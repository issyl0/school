\documentclass[a4paper]{article}


\usepackage{charter}
\usepackage{parskip}
\usepackage{fancyhdr}
\pagestyle{fancy}

\lhead{Team Australia}

\begin{document}

\title{History of Badminton}
\author{Isabell L. for team Australia}
\maketitle

Badminton was first played in ancient Greece and Egypt, when players hit an early version of the shuttlecock, made from bird’s feathers, with racquets.  It is reported to have come from an early children's game called battledore and shuttlecock.  There was an Indian version of the game in the 1700s called poona.

The British army brought this game to Britain after being stationed in India in the nineteenth century.  It became known as badminton after a party hosted by the Duke of Beaufort in 1873 at his Gloucestershire house named Badminton.  The new rules for badminton were decided in 1887 and they were simplified from the old Indian rules.

The popularity of badminton clubs has continued to rise; even in 1920 there were about 300 badminton clubs and now there are more than 9000.  Badminton was introduced to the Olympic Games in 1992 and has been popular ever since.  In 1893, the Badminton Association of England officially launched badminton in a house called ``Dunbar'' at 6 Waverley Grove, Portsmouth, England on September 13.  They also started the All England Open Badminton Championships, the first badminton competition in the world, in 1899.  The International Badminton Federation (IBF) (now known as Badminton World Federation) was established in 1934 with Canada, Denmark, England, France, the Netherlands, Ireland, New Zealand, Scotland, and Wales as its founding members. India joined as an affiliate in 1936.

\textsc{Sources:} Information from the English Wikipedia, but with sources referenced in the ``Badminton'' article and the listed sources of information were checked.

\end{document}
