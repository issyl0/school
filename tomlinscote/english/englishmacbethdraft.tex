\documentclass{article}

\usepackage{parskip}
\usepackage{fancyhdr}

\begin{document}

\title{How Macbeth changes from `worthy gentleman' to `tyrant'.}
\author{Isabell Long}
\maketitle

\lhead{Isabell Long}

`Macbeth' is a tragic play written in the 17th century by William Shakespeare; it depicts a worthy hero's downfall as he follows his ambition to become King.

At the start of the play, Macbeth, the eponymous hero, is portrayed positively as ``valiant'' and ``noble''.  He is a strong fighter and shows great bravery; he kills people with such speed that his sword ``smok'd with bloody excecution'' when he ``unseamed him from the nave to th' chops''.  King Duncan says he is a ``worthy gentleman'' confirming how highly Macbeth is regarded.  When Lady Macbeth learns of Macbeth becoming Thane of Cawdor, she ``fears [his] nature'', remarking that ``it is too full o'th' milk of human kindness'' but King Duncan thinks of him as a cousin.

Macbeth thinks evil thoughts about killing the King to become King himself after meeting the witches who prophesised he would become Thane of Cawdor and then ``king hereafter''.  The witches' prophecies caused his thoughts to turn to killing the current King.  He talks in Act 1 Scene 3 about how ``this supernatural soliciting cannot be good'' and when he thinks about killing Duncan he says ``why do I yield to that suggestion whose horrid image doth unfix my hair''.  He purposely keeps this speech away from the other characters as he does not want them to know what he is thinking; it is an aside speech.

In Act 1 Scene 4, Macbeth again hides his true feelings from those around him.  He again speaks in an aside stating ``this is a step which I must o'erleap for in my way it lies'', meaning Duncan lies in Macbeth's way to becoming King which he needs to do to fulfill the witches' prophecies which he is now becoming seriously obsessed with.  He refers to evil in the form of ``black and deep desires'' and he links to the theme of deception when he says ``stars, hide your fires, let not light see my black and deep desires''.  He uses phrases such as ``black and deep desires'' as alternatives to saying the dreaded word `murder'.

Macbeth is clearly disturbed and battling with his thoughts about killing Duncan.  He talks privately in a soliloquy about how ``if it were done when 'tis done, then 'twere well it were done quickly'' which means it is important to get the deed done quickly and he cannot dwell on it.  He talks about how ``[Duncan's] virtues will plead like angels''.  After that, he shows a sense of morality and seems to not want to kill his King when he says ``we will proceed no further in this business'', however his wife Lady Macbeth convinces him to go ahead with it, saying he should kill King Duncan otherwise she will not love him anymore.  His use of ``we will'' implies that he has made a decision not to go ahead or even think about it and he says ``we'' to take his wife into that as well.

Macbeth's hallucinations begin in Act 2 Scene 1 when he imagines a dagger with the handle towards his hand.  He cannot believe it: ``I see thee still and yet I have thee not'', he says when referring to the dagger in his hallucinations.  Macbeth is clearly doubting his own senses here and going mad and he makes reference to blood, a popular theme in this play.  That evening he kills King Duncan and fails to cope with it, breaking down.  The next morning the guards discover King Duncan stabbed to death in his chamber.  Macbeth feigns surprise and clearly wishes Duncan would awake when the guards knock, whispering ``wake Duncan with thy knocking; I would thou couldst'', regretting his actions immediately.

After Duncan's death and the guilt he feels, Macbeth turns against his friends and becomes increasingly more isolated.  He thinks more and more about the witches and feels insecure about his future and the witches' third prophecy.  He praises Banquo, saying that ``he hath wisdom that doth guide his valour'' and thinks his crown is ``fruitless'' and that it means nothing.  He then comments that images of Banquo have ``fil'd [his] mind'', showing that he is very guilty about what he has done as he killed a good friend because he stood in the way of him gaining the ``fruitless'' crown.

Macbeth became King after Duncan was killed, which he inferred before when he mentioned the ``fruitless crown'' placed upon his head.  He further torments himself and uses his position of power and authority to influence and manipulate a few of his army into killing Banquo.  He does this by convincing them that ``Banquo was [their] enemy'' and then reinforces his statement by forcing them to admit that ``both of [them] know that''.  He becomes ruthless when thinking about killing more people - ``for my own good all causes shall give way'' - but then during the banquet he admits that he is ``in blood, stepp'd in so far that, should I wade no more, returning were as tedious as go o'er'', meaning he is scared to go forwards or backwards in the pool of blood, guilt and regret that he is in because it is treacherous and he cannot do anything about it because he has gone too far.

In Act 4 the eponymous hero thinks again about the witches' prophecies and becomes even more reliant on what they tell him.  The audience realise the witches are evil in Scene 1; they tell Macbeth to ``beware Macduff'' and Macbeth decides to ``seize upon Fife'' and therefore attack Fife and Macduff's castle.  He wants to kill Macduff's wife and his children and ``all the unfortunate soul[s] that trace him in his line'' so Macduff is harmed and fears Macbeth, making him less likely to attack.  Macbeth then becomes very excited which we see through his use of exclamation marks in his speech in Act 4 Scene 1.  Macbeth now believes he is invincible because he killed both Duncan and Banquo and weakened Macduff.  He now believes he will only die of old age.

The people of Scotland now consider Macbeth to be a ``tyrant'' and dishonest and treacherous, using adjectives including ``bloody'', ``luxurious'', ``avaricious'', ``false'', ``deceitful'' and ``malicious'' to describe him and his nature.  Some of his soldiers comment that they no longer want to serve him, ``moving only in commands, nothing in love''.  Macbeth recognises at this stage that Scotland is sick, saying that ``if thou couldst [...] cast the water of [his] land, find her disease and purge it to a sound and pristine health, [he] would applaud thee''.

In Act 5 Macbeth still believes he lives a charmed life however he cannot see the point of living.  His wife dies in Scene 5 of this Act and he remarks ``she should have died hereafter''.  He repeats the words ``to-morrow, and to-morrow, and to-morrow'' and him doing so implies that he is weary and just wants to die.  He brings in the theme of light and dark when he says ``out, out, brief candle'' and he brings in the metaphor of life when he says that ``life is but a walking shadow''.  He says that his ``tale'', ``told by an idiot'', ``signifies nothing'' meaning that his life, and life in general is pointless to him.

In the last scene, Act 5 Scene 8, Macbeth faces Macduff and remarks to him that ``[his] soul is too much charg'd with blood of [Macduff's] already'' which shows morality and courage.  When the battle commences, Macbeth boasts and refers to the witches' prophecies about how no man of woman born can kill him.  A rhyming couplet creates a sense of finality just before Macbeth's throat is sliced by Macduff.

 

\end{document}
