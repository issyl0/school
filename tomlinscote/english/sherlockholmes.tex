\documentclass[a4paper]{article}

\usepackage{parskip}

\begin{document}

\title{What do we learn about the role of women in Victorian society from `The Adventures of Sherlock Holmes'?}
\author{Isabell Long}
\maketitle

During the Victorian times, women were second class citizens and their role was predominantly in the home looking after their children and husband; they were expected to dedicate themselves without question to their roles as mothers and devoted wives.  Victorian women were obedient and extremely subservient, had no power and hence were very weak.  Conan Doyle portrays women in this way in his Sherlock Holmes stories.  This essay will focus on the Sherlock Holmes stories `The Copper Beeches', `The Speckled Band', `The Man With The Twisted Lip' and `Scandal in Bohemia'.

In the first story, `The Copper Beeches', there are three women: Violet Hunter, Mrs Rucastle and Mrs Toller, however Violet Hunter is the main character and therefore I will be focussing mainly on her.  In this story, the main character is the damsel in distress, a common feature of stories in the detective genre.  Violet Hunter had been chosen as a governess in the Rucastle household, but the duties were abnormally light, the demands abnormally strange and also the pay, ``£120 a year'', was abnormally high (most governesses during that time only got £40 a year) for what she was being asked to do.  The demands included ``sit where you are told'', ``wear an electric blue dress every morning that we will provide'' and “cut your hair''.  Violet had long hair, so she was reluctant to cut it and thought that not only this demand but the high pay were most strange, so she sought out Holmes and went to seek his opinion.  Holmes investigated and it turned out that Mr. Rucastle asked Violet to be governess because she looked exactly like his daughter he was holding captive in a secret part of the house Violet was ``never to enter''.  Violet Hunter is a typical Victorian women in that she is inferior to men and her role as governess in a rich household made her subservient and naturally do as she was told.

Irene Adler in 'Scandal in Bohemia' is the first woman not to be portrayed by Conan Doyle as a traditional Victorian.  Irene Adler is young, pretty (Watson said of Irene that ``in [Holmes'] eyes she eclipses and predominates the whole of her sex''), sexy and very daring, to the point she dares to taunt the King of Bohemia with threats to send his wife-to-be a picture of a past relationship, and Holmes also thinks she is ``the daintiest thing under a bonnet'', showing he has affection for her and thinks she is pretty, which ties in with what Watson told us at the start.  She does not wear traditional clothes, sings loudly and is quite outgoing, and those are all things a typical Victorian woman would not do.

In `The Man With The Twisted Lip', Kate Whitney is the woman in typical Victorian dress portrayed as weak and helpless when she comes to Mrs Watson ''clad in some dark coloured stuff with a black veil'', in fact she is almost unrecognisable until she loses control of her emotions, and ``runs forward, throws her arms about [Mrs Watson's] neck and sobs'' and Mrs Watson lifts her veil and realises ``why, it is Kate Whitney''.  Kate was ``so frightened about Isa [her husband]'' because he had ``not been home for two days''.  She was sure he was in the opium den in a London slum, but Dr. Watson thought that ``how could she, a young, timid woman, go into such a place and pluck her husband from the ruffians that surrounded him''?  When her husband learnt that it was two days later than he thought it was, he exclaimed ``I wouldn't frighten Kate – poor, little Kate'' and this says that even her husband looks down on her and thinks she is fragile, `poor' and easily frightened.

``The Speckled Band' sees yet another typical Victorian woman as the central character and damsel in distress when Helen Stoner goes to Holmes to ask for help.  She is clearly frightened, it shows from what she is wearing and how she behaves.  Holmes sees that she is cold and shivering and she says ``it is not the cold that makes me shiver, Mr. Holmes, it is fear and terror''.  Fear and terror are strong in this quote because they show just how frightened she really is.  Her face was ``drawn and grey”, showing that possibly she hadn't slept due to fright, she had ``restless, frightened eyes'' which showed that she didn't think she was safe, and ``she was wearing a black veil'', a typical Victorian garment that indicated mourning for someone and she was wearing it because she wanted to hide herself.  Helen Stoner was hiding from her stepfather which shows that she was inferior to men, as Victorian women were, and very scared about what he would do to her if he found out she had been to see Holmes.

The male characters in Sherlock Holmes stories are presented as dominant, big, scary and they appear to overpower and frighten the women such as Helen Stoner, Violet Hunter and Kate Whitney.

Conan Doyle gives an impression of women that was representative of the time he wrote the stories in, representative of the behaviour and habits of normal Victorian women, such as they were subservient and frightened to disobey the men as the men were much bigger and had status within the house, whereas the women had no power and were expected to devote themselves selflessly to looking after their husbands and children.

\end{document}  
