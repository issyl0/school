\documentclass{article}

\usepackage{parskip}

\begin{document}

God is infinite; he has no start or end.  God is timeless.

Humans are finite.

God ``just is''.

Deists believe God created the world and has done nothing since.  If God doesn't know what happens in the future, he cannot be blamed for what happens in the future.

Undignified --> stripped of freedom.

Hinduism, Buddhism and Sikhism all like the principle of \textbf{predestination}, however Christianity is divided on it.

Pavlov's Dog = used by psychologists to demonstrate training and behaviour control --- response -- reward method.

\textit{Are we truly free in Britain?}

We have more CCTV cameras by population than anywhere else.

The behaviourist thinking suggests that human beings are subject to their social environment.  If indeed humans do not have freedom, either because their actions are already known to God before they occur, then moral responsibility becomes a problem.

\textit{How much is a human being worth?}

Human beings are worth another human beings' life, however you cannot really say.

\textsl{Then God said ``and now we will make human beings; they will be like us and ressemble us'' Genesis 1:28} --- this is about humans being made in the image of God.

\textsc{Aquinas' Just War}

The only justification in Christianity for taking life was \textsc{theory}.  It is impossible to operate today because of the rule of proportionate forces.

\textbf{Buddhism}

All about purity of mind, soul and body.  Suffering is seen as part of life for everyone; we are formed by our suffering.  `I' do not matter --- self is not important, only `we'.  Personal selfishness is not accepted.

Is it possible to understand suffering if you have never suffered yourself?

--- No, because if you haven't suffered you cannot realy empathise with the person because you haven't experienced [that type of or any] suffering.

Sentient being = something living that feels pain.

Buddhist Noble = Spiritual.

Two main branches of Buddhism:

\textsl{Theravada} --- very original and strict --- monks
\textsl{Mahayana} --- more liberal and occurs in the West more than the East.

Buddha = enlightened one.

Women cannot become a Buddha and reach enlightenment unless they are reincarnated as men.

It is important to remember that all the life of human beings with all the joys and hardships of life are a consequence of karma.

People cannot be held liable for their actions if they are not free.

\textbf{Notes from pages 88 -- 90}

Human beings have dignity that is expressed through their free actions.

If humans are not free to act then their dignity is limited, perhaps denied.

God is omniscience; God knows all the facts.

Western religions identify omniscience as a quality of God.

Aquinas argued that God was timeless.

The idea that God is omniscienct, omnipotent and omnibenevolent are features that make God far different from human beings.

The idea that God decides who recieves salvation and who doesn't at creation suggests that humans don't have free will with regard to their moral or religious behaviour.

Salvation is not earned because of something a human being does.

The psychologist Skinner argued that our ideas of dignity and freedom are widely misplaced.

This behaviourist thinking suggests that human beings are subject to their social environment rather than their will.

Human beings do seem to surprise each other by their actions, and not always in a bad way.

You can have a sense of freedom and a sense of responsibility for your actions.

In Sikhism, Buddhism and Christianity there are powerful concepts of unity and equality.  In practice however there are difficulties.

It is difficult to separate religion from the cultural context in which it exists.

All religions profess a concern for the sick and those who are typically socially excluded so by extension this can include disabled people.

The religions seem to put humans above animals.

Religions have a view of humanity and what role or purpose it has.

Humans are distinct from other [living creatures].

\textsc{christian perspectives}

The Bible teaches that people should love God and love their neighbour as themself.

In Genesis, Adam and Eve exercise free will in choosing to eat the forbidden fruit.

Human beings who live their lives following Christ become Christ-like.

St. Augustine said that the dignity of the human comes from its thinking, intellect and moral will.

Christian texts express mixed ideas about the place of disabled people but seem to appreciate diversity.

Humans may have a personal relationship with God but animals cannot.

\textsc{buddhist perspectives}

In Buddhism there is no permanent `I' or `self'.

The more attached we are to the self the more we seek gratification or satisfaction and push aside others.

All selfish action towards others is rooted in self-centered-ness.

All sentient beings are in the cycle of birth and rebirth.

The Buddhist `Noble Eightfold Path' is a profoundly ethical path to enlightenment.

A noble person is one who has been changed permanently by spiritual insights.

Each stage is progression towards the final goal of enlightenment.

Human lives are special because they have the possibility to reach enlightenment.

Change is possible for a person.

A human being is not locked into good or evil.

Change is always possible if a person chooses to go after it.

The life of all human beings are a consequence of karma.

Killing senselessly is bad.

Killing intentionally is wrong.

Killing of sentient life is worse still.

\textbf{Essay Planning Shell Notes}

\textsc{one}

\textsl{Introduction}

There are conflicting views on the value of human life.  One of the main issues is to determine when human life starts or even what is human life.

For example two religions may agree or disagree on what is human life and when it starts.

\textsl{Christianity}

Humans are made in the image of God (imago dei).  Humans are more highly regarded than animals.  Humans have dignity.  Human beings should love each other as God loves them.  Human beings are naturally evil over good --- they have to follow God.  The killing of human life destroys the being made in God's image, and therefore \textit{good}.  Christians don't say animals have no value, however.  Human life = sacred.  Couldn't cope with killing people, so `just war' = `just war' = `fair war'.  Jesus said no greater love has man than to lay down his life for another man.

\textsl{Hinduism}

Hinduism is so old we do not know where it started or who started it, therefore its concept of God is not as clear as modern religions.  They have a very profound understanding between sentient life and the soul.  Sentient life is a life form that experiences pain.  Most scientists believe that this is related to a creature that has a nervous system like humans and most animals.  Birth is the beginning of human life in Hinduism and humans are higher up the chain of birth and rebirth than animals.  The ultimate goal of all sentient creatures is the reunion with Brahman --- this makes human beings special.  Plato's view is that the body (physical) is only a vehicle for the soul --- the body itself is perishable.

\textsl{Buddhism}

The individual has a personal task to strip ego.  Accepts suffering as part of the path to enlightenment.  More rigid ethical rules --- Noble Eightfold Path.  Cannot kill anything, don't believe in killing.  Humans have the chance to reach enlightenment.  Animals have souls.

\textsc{two}

``People should always be treated equally despite/regardless of race or disability''

Religious believers accept the view that rascism and prejudice against people is always unacceptable.

It is said most people in the world are rascist/prejudiced.  Some people cannot help being rascist; they are brought up that way.  Equal but different = disability.

Buddhists and Hindus have a way of accepting disability as a consequence of past actions and therefore karma.  These religions belive in fate and predestination and can accept that disability in their current life is part of their predestined path.

Christianity will not tolerate rascism as it is a barrier to treating people equally.  Everybody in the world is equal.
\end{document}
