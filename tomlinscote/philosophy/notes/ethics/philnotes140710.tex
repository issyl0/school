\documentclass{article}

\usepackage{parskip}

\begin{document}

In Buddhism, the point of origin (start point) of life is vital.  Where does it start?

In the \textsc{palicanon}, the Buddhist scriptures, there is nothing written about rape or abortion.

Orthodox (Theravada type) --- Tibet --- Dalai Lama says ``[...] abortion should be approved or disapproved according to each circumstance''.

\textbf{Buddhistic attitudes to ethics.}

\begin{tabular}{ | p{3cm} | p{3cm} | p{3cm} | p{3cm} | }
\hline
Virtuous & Situationist & Absolutist & Utilitarian \\ \hline
Attitudes and motives matter.  An example of a virtue is being kind. & Being compassionate to the situation in hand. & Simply applying the rule i.e all killing is wrong. & Greatest good for the greatest number.  Might consider birth rate.  A Buddhist society can create a rule. \\
\hline
\end{tabular}

W.A LaFleur explores Japanese Buddhist tolerance and ritualisation of abortion.  It is a ``sorrowful necessity''.

\end{document}
