\documentclass{article}

\usepackage{parskip}
\usepackage{fancyhdr}
\pagestyle{fancy}
\rhead{23.02.2010}

\begin{document}

The term \textsc{religious experience} conjures up a wide and diverse range of images.

It can mean anything from saying a prayer to attending a service to hearing the voice of God.

A religious experience is a non-empirical (non-sensory) occurence, and may be perceived as supernatural.

It can be described as a mental event which is undergone by an individual, and of which the person is aware.

Such an experience can be spontaneous or it may be brought about as a result of intensive training and self discipline.

Recipients of religious experiences usually say that what has happened to them has drawn them into a deeper knowledge or awareness of God.

The experience that each individual has is absolutely unique and cannot be shared with anyone.

There is an infinite number of religious experiences; the main classifications are:
\begin{itemize}
\item visions
\item conversion
\item mythical experiences
\end{itemize}

\subsection*{Visions}

A religious vision occurs when an individual believes that they have seen or heard somrthing supernatural.

Three ways that an individual may experience a vision:
\begin{itemize}
\item An intellectual vision brings knowledge or understanding from God.
\item An imaginary vision where something that strengthens faith is seen in the mind's eye
\item A corporeal vision is where the figure is externally present
\end{itemize}
Sometimes dreams are considered to be visions.

\subsection*{Numinosity}

\textsc{Numinous} \textit{the feeling of the holy; includes awe, fascination, religious awareness and the smallness of self.}

Some people consider numinosity to be a mystical experience, and some people consider it to be another religious experience.

\subsection*{Conversion}

A vision often results in a conversion.

A conversion is when the effects of a religious experience are life changing.

Conversion usually results in people better understanding faith.

Religious conversion is the process that leads to the adoption of a religious attitude or way of life.

The above effects can be either permanent of temporary.

\subsubsection*{Types of conversion}

A concious and volunary experience - volitional type.

An involuntary and unconscious experience - self-surrender type.

\subsubsection*{Features of conversion}

The present wrongness in their life - their sins, for example - that they want to change.

The positive changes that they wish to make.

\subsubsection*{Examples of conversions}

Religious conversion includes a change in belief on religious topics.  This leasds to changes in a person's social behaviour within the social environment.

People speak of intellectual, moral or social conversions.

People that have a sudden conversion feel it to be a miracle rather than a natural process.

Sudden conversion is very real to those who have experienced it; God causes the conversion.

\subsection*{Mystical Experiences}

Mystical experiences are experiences where the recipient feels a sense of union with the divine.

\end{document}
