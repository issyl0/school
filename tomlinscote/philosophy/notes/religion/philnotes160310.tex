\documentclass{article}

\usepackage{parskip}
\usepackage{fancyhdr}
\pagestyle{fancy}

\lhead{16.03.2010}

\begin{document}


\subsubsection*{The conclusions of William James}

James saw people's religious experience as potential `evidence' for the existence of God.

James suggested that religious life centred on the following beliefs:

\begin{itemize}
\item the world we see around us is a part of something much bigger (something spiritual) which we consider to be very important

\item union with that spiritual `something' is our ultimate purpose

\item communication/prayer with that higher, spiritual thing is something which produces real effects and enables real work to be done
\end{itemize}

He also said that religious experience (in many forms) produces the following results:

\begin{itemize}
\item a new enthusiasm for life, often leading to profound and significant changes

\item a sense of peace and security, and of great love for others
\end{itemize}

People point out that religious experience is like emotion.  James accepts this evidence as reasonable.

James believed that regardless of which religion we study we will always find two, universal ideas:

\begin{itemize}
\item a certain `uneasiness' (i.e the feeling that there is something wrong with us)

\item the solution to this (i.e the way in which we can be saved from this wrongness)
\end{itemize}

He says that the way we find a solution is we go in search of a being `higher' than us.

James claimed that the `doorway' to the higher being\slash higher self was a religious experience in the form of mysticism, prayer and conversion.

James noted that these people who claimed to have had religious experiences seemed to be generally more fulfilled and purposeful in their understanding of the world and their place in it than those who have not had religious experiences or who are following atheism.

\subsubsection*{Sigmund Freud}

He was an Austrian psychiatrist.  He founded the psychoanalytic school of psychology.

Freud believe that people were completely material.  In more understandable terms, he believed that if we could understand everything there is to understand about the physical\slash biological side of life, we would fully understand human beings.

Freud saw religious experiences as illusions.  He thougth that they were projections of the ultimate, oldest and most profound ideas that people had.

Freud was not concerned with the truth of religious claims, so he dismissed attempts to use such experiences as evidence for the existence of an ultimate creator.

\subsubsection*{V.S Ramachandran}

Ramachandran is a neurologist best known for his work in the fields of behavioural neurology and psychophysics.

He did some reasearch and gathered important evidence liking the temporal lobes to religious experience.

He is not unwilling however to accept that it may be that God exists and that God has in fact placed the temporal lobe within the brain as a means of communicating with humans.

\subsubsection*{Michael Persinger}

Michael Persinger is a cognitive neuroscience researcher.

He agrees that the temporal loves have a significant role in religious experiences, but he also argues that religious experiences are no more than the brain responding to external stimuli.

\subsection*{Problems of verifying religious experience}

\begin{itemize}
\item individuals rather than groups undergo the experiences

\item drugs or alcohol can produce very similar effects to reported religious experience

\item religious experience is very like emotion --- it is a personal response so any form of empirical testing is useless
\end{itemize}

\subsubsection*{The objective\slash subjective distinction}

Religious experiences are regarded as subjective; a subjective experience cannot be said to be `scientific'.

Experiences happen to people and they will therefore always be open to interpretation.

\textbf{\textit{An evaluation of the argument}}

Philosophy highlights the fact that there are many ways of considering what the `truth' in a given situation might be.

The \textsl{correspondence theory} asks whether or not a particular statement corresponds to something in the `real world'.

The \textsl{coherence theory} evaluates the truth of statements by relating them to other proven truths within a given system of thought.

The \textsl{pragmatic theory} focuses on the consequences of accepting the experience; the `truth' of a statement is seen purely in practical terms.

\end{document}
