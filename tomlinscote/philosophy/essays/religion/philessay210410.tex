\documentclass[a4 paper]{article}

\usepackage{parskip}

\begin{document}

1)
a) In order to examine the claim we have to start by defining a religious experience.  Religious experiences are non-empirical experiences and some even believe them to be supernatural.  There are four main types of religious experience: conversion, visions, numinosity and mystical experience.  I am now going to look at some types of religious experience and some claims a religious believer may make with reference to this.

One type of religious experience is visions.  There are three types of visions: intellectual such as meditating on scripture, imaginary such as in dreams, and corporeal which is bodily.  Ste. Bernadette is said to have experienced a vision of the Virgin Mary, for example.

Numinosity is the sense of awe and wonder in the mind.

Conversion (from Latin `conversio' = to change) could be explained as having a religious experience whilst an atheist or having a life changing experience which draws you to God and then you convert to the religion.  An example of this in Christianity is Saul who converted to Christianity and became St. Paul after he went around persecuting religious believers and suddenly saw a vision in which God told him to go and preach Christianity, and he converted there and then.

Having examined the evidence, a religious believer would say that they have had a religious experience because of the effect of, say, conversion in that it has changed the person, like Julian of Norwich who lived a solitary life serving God.  The visions Ste. Bernadette experienced enabled her to become a Saint of the Church.

b) I will now attempt to explain to what extent I agree with the claim ``Religious experience is convincing evidence for the existence of God'', focusing on the points against religious experience.

There are many points against religious experience that show that it is not convicing evidence for the existence of God.  With regard to visions for instance, other factors such as alcohol and drugs can cause similar visions.  Religious experiences cannot be proved, they happen to anyone part of any religion, it it just a matter of faith and whether you believe the person who says they have had the experience.  Naturally, I think religious believers will be more inclined to straight away believe someone who has said they have had a religious experience than non-believers are.  Religious experiences are recounted in scriptures like the Bible like St. Paul's conversion which I mentioned in question a), but then can we believe the scriptures because after all they recount stories and people's accounts of what happened, we were not there to see the religious experience happens.  That brings me on to my other and final point - I mentioned in part a) when I defined religious experience that they are non-empirical experiences which means they are not based on the senses and they are personal, so no-one apart from the person who has had the experience can decide whether to really believe that person or have that experience themselves.  `Victims' of religious experience may find it difficult to explain their religious experience as they are non-empirical as I said, and this casts further doubt on the truth of their claim.

The bottom line is however that we are looking at a personal relationship between a religious believer.  It may be only they know the truth of their religious experience.

\end{document}
