\documentclass[a4paper]{article}

\usepackage{parskip}

\begin{document}

\title{What issuses might there be if you marry someone from a different religion?}
\author{Isabell Long}
\date{28th May 2010}
\maketitle

There are lots of issues when it comes to marrying someone from a different religion to your own (or the one you have grown up around).

Let us take the example of a British, supposedly Christian, woman marrying a Muslim man from Iran, a strict Muslim country.  Let us suppose that this British woman isn't very religious and doesn't go to Christian church in Britain, but in order for her to marry the man she loves, a strict Muslim with a strict Muslim family, she will have to convert to Islam.  The role of women in Islam is regularly disputed and it is very likely that she will face the segregation Muslim women face when she goes to Iran to see her husband's family, such as being forced to wear the hijab out of the house.  She must have understood that this is what she would have to go through, yes, before she converted!  Being a white Western woman, therefore not a `born' Muslim is going to be hard for the man's family to accept, and there is a big possibility that they won't accept it!  Also there is the point of where will they get married?  Presumably, the woman has converted to Islam before getting married so they will do so in a Mosque, but has the woman already had sex?  Islam frowns upon sex before marriage and considering the woman wasn't particularly religious before, she is likely to have already had sex before she became a Muslim.  Does, then, the chastity rule still apply as she wasn't a Muslim when she first did have sex?  It falls upon the husband and Allah to decide.

If we reverse the above example and take an Iranian Muslim woman in England marrying a non-religious British man, the man would not normally be expected to convert to Islam.  Would the woman be allowed to continue being a Muslim if she married a non-Muslim, would it be accepted or would she be shunned and outcasted by her family back in Iran and the Muslim faith?  If she does marry a non-Muslim, she would lose all contact with her family back in Iran if they don't accept it which they are unlikely to if they are strict Muslims, and her children she has with her new British husband would never get to meet their maternal aunts, uncles, cousins or grandparents.


\end{document}
