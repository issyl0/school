\documentclass[a4paper]{article}

\usepackage{parskip}
\usepackage{titling}
\setlength{\droptitle}{-0.5in}

\begin{document}

\title{Discuss the psychodynamic approach in psychology.}
\author{Isabell Long}
\date{November 28 2012}
\maketitle

The basic assumption of the psychodynamic approach is that most of human behaviour is determined by unconscious drives and childhood experience, that it has a control outside of the individual and therefore it is deterministic, unlike the behaviourist approach. Invented by Sigmund Freud, it is one of the six key approaches in psychology. It explores defence mechanisms, the unconscious mind, the psychosexual stages of child development, and the structure of personality.

Freud theorised that personality---like a person's mind---is tripartite, made up of the id, the superego, and the ego: the greedy, needful part, the moral part consciously in conflict with the unconscious needy part, and the mediator between the last two through logical reasoning, respectively.

Defence mechanisms are an important part of how humans deal with life's events. Some may choose to repress the thoughts, others choose to deflect the physically expressed emotional blows onto a safer target, usually an inanimate object like a pillow, instead of a human target, an unconscious technique known as displacement.

Freud's psychosexual stages of development are often seen as the strangest part of everything psychodynamic psychology explores. Divided into six stages, they map a child's development from birth to puperty. Despite the fact that at the time he had devised the theory before conducting research and only researched one child, in young children and adults in every day life now some of the stages can be observed, although some people could argue that evolution and modern life causes puperty to happen mentally earlier and the phallic stage is extended, and even that the latency stage may fall back due to peer pressure and sexualisation in the media.

The defence mechanism of repression, plus Freud's theory that traumatic experiences are stored in the unconscious, may be an explanation for adult mental health issues as some people cannot cope with the stress of their experiences. Indeed, not all traumatic experiences are not actively remembered---people remember deaths of close friends for example, sometimes in incredible detail. Due also to the fact that the unconscious is unobservable (as are thoughts, but cognitive psychologists manage to conduct reliable research by observing concrete behaviour, finding patterns and replicating their experiments to again find patterns), therefore unfalsifiable (may or may not exist), further doubt is cast on the authenticity of the evidence in the case studies.

Relating to both the issues raised above, some have questioned psychodynamic psychology with regard to the validity and reliability of its findings. Case studies are studies of one person, and Freud's small group consisted of merely twelve people, which in the grand scheme of the population of even a small area of the UK at that time is not the recommended general sample size, so it could not be generalised. Another argument against the theory's relevance and effectiveness is that Freud used women as case studies but based his theories, particularly the psychosexual development theory, solely on males as that was what was acceptable at the time---women were only just beginning to be seen as `worthy'---plus the fact that he was asking already mentally ill women to remember events from their childhood. They may have forgotten or adapted them to `rest' better inside them thanks to their defence mechanisms.

Despite all of these criticisms and the important fact that his research would not have been reliable if the claims that he fitted his research findings around his theories (not the other way around) are to be believed, it is important to remember that Freud is still considered as the father of modern psychology: all of the criticisms have come from further research in later years, as people just did not think in the same way in the 1900s. Some of his theories, for example defence mechanisms, still have a lot of relevance and can be observed in much the same way today.

\end{document}