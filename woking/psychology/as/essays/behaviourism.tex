\documentclass[a4paper]{article}

\usepackage{parskip,fullpage}

\begin{document}	
\begin{enumerate}
	\item{One assumption of the behaviourist approach in psychology is that people are born blank slates (``tabula rasa''), open to learning to do things from experience of life, i.e. nothing is learned before or inherited from parents.}
	\item{Reinforcement can be either positive or negative, and is defined as conditioning learning.  In this example, Haniya's mother's buying crisps when Haniya cries causes Haniya to learn that when she cries in the shop she will be bought crisps, so she repeats the behaviour in order to be able to eat crisps.}
	\item{Sunita's mind has associated hospital visits with the treatment for her medical condition, and ever time she goes into hospital for the treatment she feels sick which reinforces her belief that hospitals make her feel sick as she may not realise that the treatment is making her feel sick and not the hospital.  When she visits her grandmother in hospital, she associated hospital with feeling sick, so she feels sick even though there is nothing to feel sick about in this instance, because hospitals reinforce the feelings of sickness for Sunita.}
	\item{The behaviourist approach in psychology is one of the main approaches, along with the biological approach and four others.
	
	The behaviourist theory is a deterministic theory which assumes that people are born as ``tabula rasa'' (blank slates), open to learning from life experiences, i.e. that no behaviour is inherited, nothing is innately known.  This is in contrast to the biological approach which suggests that genetic features such as genes, which are inherited, influence behaviour.
	
	Classical and operant conditioning are two types of learning theory put forward by behaviourists.  Pavlov's experiment with dogs defined classical conditioning: learning by association with something else and natural reflexes, using conditioned and unconditioned stimuli and reinforcement.  Pavlov taught a group of dogs to salivate at the sound of a bell instead of when food appeared, over a short period of time when the two things were presented together.  Operant conditioning shows that behaviour is learned from consequences of previous behaviour, a theory thought out by Frederic Skinner.  His experiment with rats he taught to press a button to be rewarded food, a behaviour that they would obviously not normally have to do, backs this up, and of course it is an example of reward not punishment, the other consequence of behaviour defined by Skinner.
	
	Controversially, behaviourist research is often conducted on animals in this day and age as it is seen as unethical to conduct that kind of research in humans in this modern age, however little Albert is a classic example of the carefree attitude that researchers such as Watson and Rayner had towards ethics in the 1920s: they conditioned a previous fearless nine-month-old baby to fear anything white after causing a white rat to appear multiple times alongside a loud bang which made the baby jump.  As with Pavlov's dogs, Little Albert's fear of white rats disappeared after not being exposed to them for a time, thus proving that behaviour is not necessarily constant and things need to be practiced---or reinforced---to be remembered.  Referring to determinism at last, humans consider themselves to have free will, to be able to choose what they do and how they live their lives: the behaviourist approach refutes free will with its most basic principle, hence why it finds animal research easier---animals do what they need to do to survive---humans tend towards questioning everything.
	
	Although the behaviourist approach is scientific (the research findings are not open to interpretation as the experiments are well controlled), it focuses mostly on animal research due to its assumptions that animals are similar to humans and does not concern itself with mental processes or morals, so it is definitely reductionist and findings may not be able to be as easily generalised to humans given that fact.}
\end{enumerate} 
\end{document}