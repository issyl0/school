\documentclass[a4paper]{article}

\usepackage{parskip}
\usepackage{titling}
\setlength{\droptitle}{-0.5in}

\begin{document}

\title{Discuss how personality factors may be linked to substance abuse.}	
\author{Isabell Long}
\maketitle

It is often assumed that many personality disorders tend to come to the surface after a period of drug addiction, due to the change in brain chemistry causing a change in a person's moods and habits.

There is a link between increased impulsivity, extroversion, and substance (particularly alcohol) abuse.  Questions can be asked as to whether substance abuse really is the only cause of that increase.  Moeller et al.\ (2002) investigated this with cocaine abusers who were also affected by antisocial personality disorder, with two control groups.  They found that cocaine-dependent subjects with antisocial personality disorder were more impulsive and aggressive than their control group, but the non-cocaine-dependent antisocial personality disorder sufferers were equally as aggressive compared to their control group, so there was no conclusive evidence for the effect of substance abuse on increased aggression levels when a personality disorder was already present.

Stress was found to also be a contributory factor to substance abuse relapse (Sinha, 2001).  People susceptible to stress will not always go on to become substance abusers, of course, but when at low points people tend to take comfort in what they may know as wrong but easy, especially if previously experienced, and so may turn to drugs (either illegal or legal, for example consuming caffeine or alcohol in excessive quantities), leading to adversely affecting the lives of people around them and sinking even lower.

A lot of research into substance abuse and personality tends to focus on alcohol abuse, as it is the most common and therefore seemingly easiest to conduct research into, for example there is strong evidence for hereditary personality traits contributing to alcohol abuse in later life (Cloninger et al., 1998): however this study did not take into account any other personal circumstances which may have led to the abuse, however the number of children studied who had the predefined personality traits and ended up alcoholics was large enough for personality to be seen as a substantial cause.  Medical professionals have in history been found not to pick up on the signs of particularly alcohol-related substance abuse and associated mental health problems and therefore personality disorders (Coulehan et al., 1987), leading to many being undiagnosed and untreated and the problem getting worse.   Therefore, personality factors linked to substance abuse is a broad topic which the research does not focus on all of.  Also, with even medical professionals not picking up on all of the signs, potentially even in this day and age when technology is seen by some as a hindrance and doctors have hundreds of patients, this all could lead to unclear representations of the effects of substance abuse and a potentially misinformed general public and a lack of trust.
\end{document} 