\documentclass{article}

\usepackage{parskip}
\usepackage{a4}

\begin{document}
	
\title{Discuss the psychodynamic theory of crime.}
\author{Isabell Long}
\date{October 12, 2012}
\maketitle

The psychodynamic theories stem from Freud, and many aspects of them can be applied to crime: maternal deprivation, the superego, and defence mechanisms.

Three different types of superego may contribute to offending, according to a study by Blackburn in 1993: weak, deviant, and over-harsh.  Those with the latter type of superego may actively seek out opportunities to be punished due to their need for it, whereas those with deviant and weak superegos are said to have been affected by the absence of a same-sex parent and therefore moral values during childhood, however to counter this theory, it is true that plenty of people born into single parent families grow up law-abiding.

The above is similar to maternal deprivation, another part of the theory devised by John Bowlby in his experiment in 1951.  Bowlby found that if a child is deprived of ``a stable and loving maternal relationship'' between the age of 0 and 2 they are more likely to go on to commit crimes or at least be juvenile delinquents.  However, some dislike this study as Bowlby's experiment looked back at child development, he didn't study the children periodically from birth, leading to potentially invalid results unable to be generalised.  Therefore, it could be said that not all of those experiencing maternal deprivation will go on to commit crimes, and equally it would not be a prerequisite, but in support of Bowlby's methods it would have probably been quite difficult to find maternally deprived children and study them from birth before 1951 when psychological research was potentially an alien concept to most people, so he had to rely on historic accounts and truthfulness of participants.

Defence mechanisms are present in criminals and those who have not committed crimes alike.  Use of the subcategories is not restricted to criminals.  Take denial, for instance, it is particularly usual for a child to deny having eaten some chocolate before dinner time, but that is not a crime per se, just boundary testing with parents.  After this, the child may punch a pillow, a technique known as displacement.  This act on a truly inanimate object is different to a person riled by a work colleague after a long day stabbing a passer by on his or her way home---even someone considered inanimate by the fact that the killer did not know and so did not have any feelings towards them.

Overall, the psychodynamic theories of crime do not account for everything related to criminal behaviour, as has been explained: learning theory and physiological explanations such as the somatotype theory of body type may have some merit.  Equally though, it could be argued that the latter theory is just as vague and has no real basis for defining who will or will not offend, especially with Sheldon's lack of use of the formal criteria for delinquency---he may have put delinquents in the non-delinquent group and vice versa.
\end{document}