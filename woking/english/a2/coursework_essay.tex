\documentclass[a4paper]{article}

\usepackage{parskip,fancyhdr,titling}

\setlength{\droptitle}{-1in}
\setlength{\headheight}{15.2pt}
\pagestyle{fancy}
\lhead{Isabell Long}
\rhead{A2 English Language \& Literature coursework}

\begin{document}
\title{Using an integrated approach, examine the presentation of blindness in Shakespeare's King Lear. Make a comparison with Bennett's Soldiering On.}
\author{Isabell Long}
\maketitle

King Lear is a tragic Shakespearean play which explores the descent into madness of the eponymous hero as he naively abdicates the throne and his kingdom in preparation for retirement. Soldiering On is a monologue by Alan Bennett which explores the life of Muriel after her husband's death and how she is subsequently devalued by her son and ends up living in sheltered housing, stripped of everything she had that made her wealthy and stand out.

The theme of blindness is strong in both King Lear and Soldiering On, as are the themes of sibling rivalry and family destruction.  Gloucester, for example, the father of Edmund the ``illegitimate'', a ``bastard'', and Edgar the ``legitimate'' is in act one blind to the fact that Edmund is lying in order to gain control of his estates when he dies, and as intended by Edmund's deceipt assures Edmund that as he is a ``loyal and natural boy'' he will ``work the means to make thee capable'' for inheritance, a ghastly prospect for the reader to consider as Edmund is a machieval.

Also in act one, King Lear abdicates his kingdom to his daughters and their husbands based on a love test, and he expects it to all go smoothly, i.e. that he will ``crawl unburdened to death [\ldots] future strife may be prevented''. Just the fact that he expresses himself like this and thinks that he could ``crawl'' with no problems whatsoever is early proof of his blindness. This love test ends badly both for Lear, who gives up his divine right of Kings, and for Cordelia who is ironically the most honest of the three in expressing her love. Lear feels insecure as he does not receive the same fake flattery from her as he does from Regan or Goneril (the latter of whom gushed ``I love you [\ldots] dearer than eyesight, space and liberty'', which is quite obviously hyperbolic), and as a result he refuses to listen to Cordelia when she speaks what the reader sees through dramatic irony as absolute sense: ``[w]hy have my sisters husbands if they say they love you all?'', so he is metaphorically blind. Cordelia in act one scene one is banished to live with the King of France as Lear screams ``here I disclaim all my paternal care'', a rash statement said in the heat of the moment that he later regrets (``I loved her most'', he painfully tells Kent, the superlative ``most'' suggesting this regret), and he banishes anyone who tries to help him see sense such as his long-standing friend Kent, thereby destroying his family and increasing his levels of blindness as he fails to see who really wants to help him as he just believes grandiose lies.

Monologues are obviously a different style to plays: still spoken, just much more intimately. In the production of Soldiering On in Alan Bennett's Talking Heads collection, the atmosphere is made tense and at points sad with close-up shots or long shots of Muriel, when she is sat lonely in her bedroom with boxes all around her ready to move out, for example. She is there depicted as the very tragic main character, with a hamartia (flaw) that the philosopher Aristotle said in his work Poetics that any tragic play had to have. She is metaphorically blind. Contrarily to Lear, Muriel does not go visibly crazy or die in the end: Muriel stays stubborn in the face of being reduced to borrowing cassettes from the library for entertainment, not wanting to use Meals on Wheels. In fact, she is offended when she offers voluntary help to the Council delivering the aforementioned Meals on Wheels and it is assumed that she is asking to be served by the initiative. Her insecurities are not helped when she realises and indirectly and carefully infers (``[the psychiatrist] wanted to know [about] when [Margaret] was little [\ldots] bloody psychiatrist!'') to the listener that her daughter Margaret's mental illness is due to her husband having abused her as a child, and she turned a blind eye, something that will stay with her now she does not lead such a prosperous and busy life that used to enable her to forget about the fact that she then was purposely blind to it because she could not have spoken ill of her husband.

Gloucester once again shows off his ignorance and blindness by not recognising his own son Edgar who, forced into exile by Edmund's evil tales of letters written, is forced to morph into `Poor Tom', a peasant, covering himself in leaves and living in the woods, finding himself on the hill in a hovel on the night of Lear's psychoticism. He joins in with the madness and the singing and the back to front days: ``so, so, so, we will go to supper i'the morning'', where ``so, so, so'' being an epizeuxis, with the ``s'' beginning ``so'' also being an example of sibilance, which adds to the sound of the sentence and emphasises Lear, Kent and the Fool's madness, despite the Fool usually being Lear's rock---sentences do not usually start with ``so''. Shakespeare's use of the ``i'the'' construct refers back to the time in which the play was written, and Shakespearean English, like ``thee'' that was quoted earlier in this essay. In the Folio edition of the play, a few pages back Shakespeare is purported to have originally written ``[c]essez!'' as an exclamative and something that Edgar says. This displays Edgar's level of education, as he is still playing Poor Tom at this point and Poor Tom had no education so would not even have knowledge of French, let alone know which context to put it in. Therefore, Gloucester's blindness is again emphasised, and some may even argue that his blindness is not just mental but also aural as it is shocking that he did not recognise his own son's voice, even when he had previously banished him and ordered him killed after Edmund's nasty and fake revelation (``[Edgar attempted to] persuade me to the murder of your lordship'', ``your lordship'' being a respectful and endearing term that Gloucester expected of his good son) on line sixty-two in scene one in act two: ``not in this land shall he remain uncaught [\ldots] bring the murderous coward to the stake'', ``coward'' in this sentence implying that he was too feeble to face up to what he had done and ran away, not something that Gloucester would have brought him up to do. Gloucester's blindness mirrors Lear's, and Edgar's naivety makes him vulnerable to Edmund's tricks.

Lear is aware he is going mad on the hill for example in act three (``his wits are gone'', declaratively states Kent having returned in disguise as a servant as he could not bear to see Lear like he was), and struggles and begs for it to stop, with his Fool, his `rock', distraught by his side. Shakespeare's use of metaphors is very evident in Lear's hill soliloquy: ``blow, winds, and crack your cheeks''---winds ``blow'', yes, but they don't have ``cheeks'' apart from in cartoons, and the wind does not ``crack''. However, the metaphors show how ferocious the wind is and this can be compared to the terrible state of Lear's mind. Just afterwards, Edmund is given a soliloquy in which he states that ``the younger rises when the old doth fall'', inferring that he wishes to break Lear, potentially executing him through causing him to lose ``no less than all.'' Shakespeare here could have just written ``everything'', but the use of language implies the status of Lear and Edmund and that he definitely wants to break him: ``everything'' does not have the same connotations as ``no less than all'': it is much less grand.

In act three scene seven on line eighty-four, Gloucester has his eyes gouged out by Regan, Lear's evil daughter who now has immense power, but is instantly metaphorically sighted and realises Edmund's wrongdoings. He implores him to ``quit this horrid act'', with the adjective ``horrid'' implying that Edmund is evil and a traitor, which the readers or watchers have all known throughout the play due also to dramatic irony. The quotation``I stumbled when I saw'' emphasises Gloucester's odd kind of blindness and increasing the sense of irony, with ``stumbled'' and ``saw'' being examples too of sibilance.

Compared to Lear, Muriel lives her life as best she can, accepting getting older and her changing circumstances due to her husband's death, as she believes---as every parent does---that in later life his or her children will always care for her as their parents dedicated their lives to them, not trick them. In this respect, both Muriel and Lear are very similar and neither is better off than the another in the end, although it could be argued that Muriel is better off because she is still living and has a chance to make amends, and she does not see herself as depressed and alone. Muriel commented right at the end ``I'm not that sort of woman'', a seemingly desperate, affirmative statement of which she is trying to convince herself, which increases the reader's levels of pity and her tragic hamartia as she is most definitely blind to what is really happening to her: status degradation and essentially living dead.

\begin{center}
	\textbf{Final word count: 1548 words.}
\end{center}

\end{document}