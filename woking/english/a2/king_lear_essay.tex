\documentclass{article}

\usepackage{parskip}
\usepackage{fancyhdr}

\pagestyle{fancy}
\lhead{Isabell Long}

\begin{document}

King Lear is a tragic Shakespearean play which explores the descent into madness of the eponymous hero as he naively abdicates the throne and his kingdom in preparation for retirement.  Soldiering On is a monologue by Alan Bennett which explores the life of Muriel after her husband's death and how she is subsequently devalued by her son and ends up living in sheltered housing, stripped of everything she had that made her wealthy and stand out.

King Lear abdicates his kingdom to his daughters and their husbands based on a love test, which ends badly for Cordelia, ironically the most honest of the three in expressing her love.  Lear feels insecure as he does not receive the same fake flattery from her as he does from Regan or Goneril (the latter of whom gushed ``I love you [\ldots] dearer than eyesight, space and liberty'', which is quite obviously hyperbolic), and as a result he refuses to listen to Cordelia when she speaks what the reader sees through dramatic irony as absolute sense: ``[w]hy have my sisters husbands if they say they love you all?''  Cordelia in Act 1 Scene 1 is banished to live with the King of France as Lear screams ``here I disclaim all my paternal care'', a rash statement said in the heat of the moment that he later regrets (``I loved her most'', he painfully tells Kent), and he banishes anyone who tries to help him see sense, such as his long-standing friend Kent.

The theme of sibling rivalry is strong in both King Lear and Soldiering On, as is the theme of blindness.  Gloucester, for example, the father of Edmund the ``illegitimate'', ``bastard'' and Edgar the ``legitimate'' is in Act One blind to the fact that Edmund is lying in order to gain control of his estates when he dies, and as intended by Edmund's deceipt assures Edmund that as he is a ``loyal and natural boy'' he will ``work the means to make thee capable'' for inheritance.
 
In Act One, the only person who Lear can really turn to turns out to be his Fool: the person who makes him laugh but who also is brutally honest in telling him in roundabout ways that he has done things wrong.  The use of humour injects different personality into both the text and the stage production, enabling the reader or audience to really connect with Lear's main hamartia and feel very sorry for him as the tragic hero being taken advantage of by his own family.

Monologues are obviously a different style to plays: still spoken, just much more intimately.  In the production of Soldiering On in Alan Bennett's Talking Heads collection, the atmosphere is made tense and at points sad with close-up shots or long shots of Muriel, when she is sat lonely in her bedroom with boxes all around her ready to move out, for example.  She is there depicted as the very tragic main character, with the same hamartia (flaw) as Lear: trust and the naivety of expecting honesty from everyone.

Overall, the hamartia of all three parents examined in this essay is common in the world---no parent ever wants to believe that their children are dishonest and doing them harm when they themselves are powerless to react because they are so emotionally drained.

\end{document}