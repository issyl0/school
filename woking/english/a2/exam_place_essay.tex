\documentclass[11pt,a4paper]{article}

\usepackage{parskip,fancyhdr,titling}
\setlength{\droptitle}{-1in}
\setlength{\headheight}{15.2pt}

\begin{document}

\title{Compare and contrast how a sense of place is created in texts A--C.}
\author{Isabell Long}
\maketitle

Text A, Tintern Abbey, text B, Under Milk Wood, and text C, a page from the Indian Ministry of Tourism's website, each convey a strong sense of place to their readers, and have portrayal of place in common with each other.

Tintern Abbey starts by giving the reader a sense of time passing, that of ``the length of five long winters''. Under Milk Wood begins, clich\'{e}d, ``at the beginning'', and text C dives straight in by making the reader think of ``amazing diversity'', the premodifier amazing instantly bringing to the reader's mind all of what they think India is---bright and colourful, for example.

As has come to be expected from descriptive and persuasive texts, imagery plays a huge part in setting the scene and guiding the reader on his or her journey. In Tintern Abbey, ``orchard-tufts [\ldots] are clad in one green hue'', ``one green hue'' implying that they don't stand out but are lost in ``wild [\ldots] seclusion''. This parallels Wordsworth's narration as he appears lost in exploring Tintern Abbey's grounds and its spectacular spring scenery.

Similes and metaphors are often used in descriptive texts, and these three are no exception. In Tintern Abbey, on line 24, Wordsworth writes that ``these beauteous forms [\ldots] have not been to me as is a landscape to a blind man's eye''. Through this simile, he makes the reader think about the logistics and obvious impossibility of a blind person appreciating a landscape in the same way that a sighted person does. This startling thought may lso cause the reader to doubt the rawness of Wordsworth's feelings and appreciation of the ``beauteous forms'' of the landscape's hedge-rows and caves. Maybe, in fact, a blind person would appreciate the other aspects of being in that place, i.e.\ the noises or the smells, which Wordsworth does not at all mention, due to not being able to see so having heightened senses and a potentially greater appreciation---even on his or her second or third visit. However, some may argue that the previous point of not being `blind' to the new experiences is what Wordsworth is trying to say by ``have not been to me'', i.e.\ that he does in fact notice the new things, as he then goes on to gush about how being in the countryside or having thoughts of it provide him with ``tranquil restoration'', the adjective ``tranquil'' implying immense peace and quiet and escapism from his daily life, all caused positively by nature. Under Milk Wood also includes the theme of blindness through anthropomorphising houses as moles, assumingly so as to make the reader imagine their small dimensions but efficiency and usefulness.

The Ministry of Tourism for India's writing is deliberately geared towards making people want to visit India. It uses heavy imagery in almost every sentence, with many adjectives such as ``dazzling'' implying immense, unbelievable beauty. Sibilance such as ``subcontinent is sizzling'' makes sure that people know just how unbearably hot it can get. This is followed by a nod to the ``spectacular retreats of the heady Himalayas'', mentioning a main attraction across the very north of India that people have undoubtedly heard of, ``heady'' implying mind-blowing great heights and a pleasing coolness apt to ``retreat'' to after the intense humidity and general hecticness of Indian city life that people will imagine from the country's portrayal in the media. The last sentence is deliberately punchy: ``a place of infinite variety'', which echoes in fewer words everything that the article previously says, and due to the genre is more upbeat and positive than Tintern Abbey. Indeed, Under Milk Wood which ends on a whispered, dull, ``all the people [\ldots] are \textit{sleeping now}'', implying a dead town with not much life to it, but a nighttime world stunningly beautiful in an odd, serene way, described as ``starless, bible-black [and] silent'', with the plosive alliteration ``bible-black'' just emphasising the emptiness.

The way in which the author creates a sense of place is strong in every text. Overall, it seems to me that Tinern Abbey is the most evocative as Wordsworth was a Romantic poet and so writing about and describing the beauty of nature and setting as he did was natural of the era, and it gives the reader an honest and thought-provoking mental image of the Abbey and its grounds and surroundings, whereas Under Milk Wood is dour and depressing and the Ministry of Tourism for India's website copy may not be totally honest about the stupendous beauty of India as it is trying to make money from visitors. 
\end{document}