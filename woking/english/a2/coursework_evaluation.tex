\documentclass[11pt,a4paper]{article}

\usepackage{parskip,fancyhdr,titling}
\usepackage{titling}

\setlength{\droptitle}{-1in}
\setlength{\headheight}{15.2pt}
\pagestyle{fancy}

\lhead{Isabell Long}
\rhead{A2 English Language \& Literature coursework}

\begin{document}
\title{Evaluation}
\author{Isabell Long}
\maketitle

I have written a persuasive speech entitled `Why you should be enthusiastic about teaching programming in secondary schools', inspired by Michael Gove's reforms of the ICT curriculum and because the subject area is a personal passion of mine.

At the beginning, it is clear who my audience is as I have directly addressed them: ``you, committed secondary teachers'', with the personal pronoun ``you'' giving them a sense of importance and the adjective ``committed'' inferring that they should be committed to their jobs and passionate if they are not already, which they probably would be if they were at the conference that this speech is intended to be given at.

Further on, I have used some subject-specific lexis (``Raspberry Pi [the credit-card sized computer]'', ``National Science Learning Centre'') to further enthuse the audience and cause them to remember to search for these terms or organisations afterwards, adding to their enthusiasm. The notion that programming ``will be seen as a chore'' for some teachers is a delicate one, but an issue that needed raising as indeed some teachers do ``struggle to keep up with the rapidly growing potential of their students.'' This is a declarative sentence, but it was intended to strike a chord and appeal to and involve my audience, cause them to stand up and not get left behind.

Short, imperative sentences have been extensively made use of, such as when talking about the ``Raspberry Pi-related projects'': ``Try them. Be enthused.'', challenging the audience to go away and regain interest in their subject.

After referring again to my audience by directly addressing them, the strong metaphor ``did the publicity fall on blind eyes or deaf ears?'' is also a rhetorical question. This adds to the punchiness of the speech and the relevance to the teachers---they would not be teaching if they were blind or deaf, but their inboxes are overflowing. How many emails would they have ignored?

Making reference to the Secretary of State for Education ensures some formality within the speech as an authoritative figure in Government has evoked similar thoughts to mine, reforming the curriculum from scratch. With this mention of Michael Gove, teachers will potentially be less sceptical of the messages and more inclined to think of them as something really serious and so act on it.

After the passionately evoked sentences of incrementum ``can be [\ldots] would be [\ldots] should be understood'', a syndetic list was used to link programming concepts with English and maths: ``syllables, words, sentences, digits, numbers'', and a real world example of primary school teaching methods with the ``turtles moving around the floor'' that were programmed seeks to make teachers' eyes light up with nostalgia: the repetition of the word ``remember'' enforces their remembering (or imagining) of these moments, especially with the imperative short sentence once again---``[r]ekindle that''---which implores them to empassion the teenagers that they teach. 

Exclamative sentences take the stage to conclude, implying heartfelt passion and again relating back to the audience of conference delegates: ``[s]o, teachers, unite! Rise to the challenge[!]'' with those imperative verbs adding to the passion of this enthused conclusion. I envisaged a more sombre tone for the last sentence as it expresses the key purpose of the speech: ``do[ing] pupils justice.''

\begin{center}
	\textbf{Final word count: 542 words.}
\end{center}

\end{document}