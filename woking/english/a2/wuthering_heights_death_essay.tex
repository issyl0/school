\documentclass[a4paper,12pt]{article}

\usepackage{parskip,fancyhdr,titling}
\setlength{\droptitle}{-1in}
\setlength{\headheight}{15.2pt}

\begin{document}

\title{How does Bront\"{e} present the theme of death in Wuthering Heights?}
\author{Isabell Long}
\maketitle

Death is an important theme in Wuthering Heights, given its Victorian and isolated setting. Many of the characters die throughout the book and it appears to be a welcome step on life's ladder as it frees them from the problems and social politics of living at the Heights.

Towards the beginning of the novel, Catherine's father Mr.\ Earnshaw dies. It could be interpreted that Catherine causes her father's death, with him asking her ``why canst thou not be a good lass'' as a rhetorical question as he reflects on his children's upbringing, the archaic language being typical of the Victorian era and ``lass'' being Yorkshire dialect for ``girl''.

Hindley returns for the funeral. He was sent away to college to avoid him tormenting Heathcliff, and found himself a wife. Nelly Dean, the servant, describes the wife as ``shivering, clasping her hands'', the dynamic verbs ``shivering, clasping'' implying panic and a lack of confidence, and she ``felt so afraid of dying!''. This is an exclamative sentence with the intensifier ``so'' emphasising the sheer terror she was consumed by. It is possible that the wife epitomises Bront\"{e}'s fears and weaknesses, as Bront\"{e} herself was frequently taken ill and her sisters also died very young, as was typical of the Victorian era due to poor sanitation.

The reader is not told about Hindley's wife's death, but Hindley's death on pages 187--188 is spectacular. Much description of the drunken state he was in occurs and Heathcliff gets his revenge for his previous mistreatment and bullying by degrading him with his use of the simile and anthropomorphic expression ``drunk as a horse'', going on to suggest that he does not deserve to be buried in a cemetery because his alcohol abuse was akin to ``suicide'' and he was not high enough up the social ladder due to this. Heathcliff wishes him to be buried namelessly at the crossroads, somewhere non-descript. Nelly makes a case for a proper burial, however, and Heathcliff relents due to wanting to keep her and everyone else on-side as he will inherit the Heights and Thrushcross Grange and become a rich and powerful man, moving high up the ladder of social class. Heathcliff is portrayed as sadistic as he does not console Hareton but suggests metaphorically that ``one tree will grow as crooked as another'', i.e. that with the same harsh treatment, Hareton will become like Heathcliff: bitter and twisted.

Catherine's death (as opposed to young Cathy's) in volume two chapter one spirals from feverish madness and ``violent, unequal throbbing of her heart'' to her dying ``quiet as a lamb [\ldots] sinking to sleep'' after she gives birth to Edgar Linton's daughter who is also named Catherine but known as Cathy. The simile ``quiet as a lamb'' shows the reader the tenderness that she still possessed from birth that had just been obscured by her lust for Heathcliff and hatred of being enclosed in the Grange. Before she dies, Heathcliff visits her and she exclaims ``I shall die! I shall die!'' These are two exclamative imperative sentences which illustrate her self-pity. This is a self-fulfilling prophecy as she is convinced that she is going to die and therefore makes no effort to fight it, eventually giving in to the peaceful release from the stress of unfulfilled love and life. The exclamation, repetition and begging Heathcliff to stay shows her descent into madness brought on by the fever, in stark contrast with her eventual, silent death. It is as if she was waiting to see Heathcliff again before she passed away, knowing all along her fate.

Overall, death in Wuthering Heights occurs frequently and although it is feared, people know that it is inevitable and when times get tough they look forward to dying as it provides a release from the torment of living at the Heights, the social class divide, and friendships and unfulfilled relationships, however people do live on in the afterlife according to the Victorians. This is portrayed by Bront\"{e} after Catherine dies, for example, Heathcliff goes as far as digging her up in order to feel closer to her, as she ``torments'' him, and then cannot wait to die to join her and love her the way they could not in person due to social constraints and feelings of inequality and unworthiness.

\end{document}