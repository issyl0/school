\documentclass[11pt,a4paper]{article}

\usepackage{parskip,fancyhdr,titling}
\usepackage{titling}

\setlength{\droptitle}{-1in}
\setlength{\headheight}{15.2pt}
\pagestyle{fancy}

\lhead{Isabell Long}
\rhead{A2 English Language \& Literature coursework}

\begin{document}
\title{Why you should be enthusiastic about teaching programming in secondary schools.}
\author{Isabell Long}
\maketitle

I firmly believe that programming should be taught to children in schools, by you: the same, committed secondary teachers that taught word processing and spreadsheet skills in previous years. Its use practically in the classroom for projects would provide a new perspective and the potential for more interactive teaching styles. Ultimately, making these machines do what you want them to do with instructions is fun and a great way for children to learn in the classroom and independently.

There is the danger that it will be seen as a chore for you, the teachers, who may not see potential in their students as you struggle to keep up with the rapidly growing passion of their students and prefer to stick to the mundane, more simple elements of ICT. This is what the National Science Learning Centre is trying to guard against by gathering resources in order to run new Raspberry Pi-related projects and teach the teachers. Try them. Be enthused.

Problem solving skills are important---think about the maths curriculum as evidence for that---as is agile development (rapid prototyping of projects). Organisations such as Apps for Good or Young Rewired State teach or provide an outlet for already skilled kids respectively, and these are brilliant initiatives. Their problem each year is recruiting kids and kindling the teachers' interest in the first place. Teachers in the room, did you know about these initiatives, or did the publicity fall on blind eyes or deaf ears?

If the more savvy of you are wondering which programming languages to teach, there are a plethora! My advice: make sure it is cross-platform---not everyone has a Microsoft Windows PC to run Visual Basic code on, some of us use Mac OS X---think about your media studies or art departments! Python or Ruby are good choices of languages for older teenagers, although the former has also been picked up with the help of the the credit-card sized componenty computer known as the Raspberry Pi,  by seven year olds in primary school!

The Secretary of State for Education's ICT curriculum proposals follow Piaget's main stages of development, however my previous comment showed that children should not be underestimated! Emma Mulqueeny has written extensively personally with reference to the Young Rewired State, that `Year Eight is Too Late' to get kids into coding. Abstract concepts such as variables and the different data types can be understood, would be understood, and should be understood! Indeed, the latter ties in with English and maths---syllables, words, sentences, digits, numbers.

Do any of your students remember the turtles moving around the floor when they directed them? Do they remember feeling amazement on their faces when it worked? Rekindle that. Don't just settle for word processing: that's boring. There are a multitude of resources out there just waiting to be found, developed by world leading universities such as MIT's Scratch, and online interactive courses like Codecademy which are constantly growing.

So, teachers, unite! Rise to the challenge of coding, do your jobs for your own satisfaction and personal development, but more importantly for the love of doing your jobs and having something new to teach rather than the mundane! Do it, before you no longer do your pupils justice.

\begin{center}
	\textbf{Final word count: 541 words.}
\end{center}

\end{document}