\documentclass[a4paper]{article}

\usepackage{parskip,fancyhdr,titling}
\setlength{\droptitle}{-1in}
\setlength{\headheight}{15.2pt}

\begin{document}

\title{How is the theme of marriage presented in Wuthering Heights?}
\author{Isabell Long}
\maketitle

Marriage is a central theme of Wuthering Heights. The novel being written in the Victorian era almost guarantees this as marriage was very important to the Victorians and indicated a women's worthiness and her husband was her status symbol, a way to move up the ladder of social class.

One of a woman's functions in the Victorian era was to marry and have children to continue the genetic line. Due to unsanitary conditions, many women died in childbirth and this was a known risk. Heathcliff empregnated Isabella, for example, and she died in childbirth, leaving Linton without a mother. The reader knows in general that Heathcliff did not love Isabella, just sought revenge on Catherine for marrying Edgar. From a more modern and feminist perspective, it could also be argued that potentially in general society men did not actually love their wives if they were willing to put them through the pain and risks of childbirth knowing that their child would almost certainly end up motherless and it would be down to the maids to care for him or her.

Similar happened to Hareton Earnshaw when his mother and father died as happened to Linton. Heathcliff sought revenge for Hareton's father Hindley's teasing and bullying of him by snarling ``we'll see if one tree won't grow as crooked as another, with the same wind to twist it!'', Bront\"{e} using a simile here to evole the unfair persecution of the young, distraught child for something he was not responsible for when he didn't have a mother around to defend him.

Heathcliff is haunted by Catherine all the way to his death as his exclamation of ``I am surrounded by her image!'' evokes that fact. ``Surrounded'' is the expression of his feeling of being overwhelmed. His pained use of the personal pronoun ``her'' seeks to degrade her in his mind, make her less of a real ghost, reduce her to something that he should not care about. He seeks revenge on almost everyone in an attempt to bring her back as he feels that they are destined to be together. He resents having gone away and not having sought Catherine's love and hand in marriage due to her feeling degraded by his inability to climb social classes and be a man. He goes away to do just that and shocks everyone with his presence when he returns three years later---Nelly was ``amazed, more than ever, to behold the transformation'', but still remarked that a ``half-civilised ferocity lurked yet in the [\ldots] eyes full of black fire''. The metaphor ``eyes full of black fire'' mirrors the start of the book when Heathcliff is introduced to the Earnshaw family, disliked on first impressions and described as a ``gypsy brat'', which just goes to confirm that however much we try to change ourselves in our life, our roots always show through in some way and our eyes are generally the true purveyors of emotion.

Marriage in Wuthering Heights starkly contrasts with the death portrayed in Truman Capote's In Cold Blood in general principle. With regard to the amount of misery that both events cause to everyone affected by them, marriage is about as pleasant as the description of the brutal murder of the Clutters as the characters in Wuthering Heights rarely get what they really want when they marry. Heathcliff, for example, married Isabella to spite Catherine as she married Edgar because she thought that ``marrying Heathcliff would degrade [her]'' as otherwise she would not have been able to move up the social classes. This ruined the lives of all four people concerned as Catherine was lusting after Heathcliff despite Isabella being treated abysmally by him as his wife, and Edgar retreated to his books leaving Catherine feeling more alone and unfulfilled than ever in a marriage of pure convenience.

The only marriage which is considered in the book to show real love between the two people is Cathy and Hareton's. This marriage is one of love and stemmed from a good friendship, it wasn't forced like Cathy and Linton's. It mirrors Catherine's relationship with Heathcliff at the beginning of the novel with the slight social class deviance and Catherine being daring enough to run around with him. Cathy nurtures Hareton and teaches him to read, rewarding him with ``kisses'' when he reads things right, or ``pulling his hair'' when he does not. This leads Hareton to regain the heart and softness that Heathcliff had evily forced out of him. (Heathcliff's influence is shown when he throws a stone at Nelly when she visits, leading her to brand him with the adjective ``vicious''). At this point, Heathcliff is weak and gives up seeking revenge. He feels an affection towards their relationship, as if he feels it mirroring what him and Catherine could have had. Their marriage coming right at the end of the book leaves the reader with a sense of hope for Wuthering Heights and Thrushcross Grange, a hope that things will get better and the houses will be rid from their isolation and revenge-seeking ways of Heathcliff.

\end{document}