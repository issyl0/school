\documentclass[a4paper,12pt]{article}

\usepackage{parskip,fancyhdr,titling}
\setlength{\droptitle}{-1in}
\setlength{\headheight}{15.2pt}

\begin{document}

\title{Explore the importance of place in Wuthering Heights.}
\author{Isabell Long}
\maketitle

Emily Bront\"{e}'s Wuthering Heights is a novel that explores the lives and incestual relationships of two families who live in two houses in the rural Yorkshire moors---Wuthering Heights and Thrushcross Grange---from the point of view of the curious new tenant of Thrushcross Grange being told the history of the houses by a loyal servant. The novel includes many references to local setting and place. It is well known that Bront\"{e}'s life was very insular, with many of her family dying young, and her unworldliness is presented in the characters' movements: nothing is said about where they travel to when they return, for example.

When the outsider and initial narrator Lockwood comes to Wuthering Heights as he cannot find his rented property in the coming storm, he comments that he ``paused to admire the quantity of grotesque carving lavished over the front [\ldots] door'', ``grotesque carving'' meaning the gothic elements so common to Victorian society included on houses as an expression of status. Here he uses the adjective ``lavished'' to imply that the Heights must have had rich owners to have afforded such carvings. From Lockwood's use of ``grotesque'', his use of ``admire'' could be interpreted as sarcasm which adds to his immediate dislike of the north. This contrasts with his exclamative sentence on the first page of the book which seems very genuine when he is riding his horse across the beautifully green Yorkshire moors---``[t]his is certainly a beautiful country!'' Lockwood initially identifies with Heathcliff, thinking of him as a ``capital fellow'', but as the book progresses he realises how repulsive he can be, observing in the first chapter that he ``forms a singular contrast to his abode and style of living'', due to the Heights' internal appearance as ``belonging to a homely, northern farmer'', i.e. someone welcoming and warm who extends a handshake to guests, which Heathcliff with an ``erect and handsome figure'' is the opposite of.

In chapter four, the narration is taken over by Nelly Dean, the servant, telling stories of family life between the two houses. Her narration morphs into young Heathcliff's as the time when young Catherine and him escaped to Thrushcross Grange is recounted. Heathcliff and young Catherine were portrayed as mesmerised by Thrushcross Grange's ``splendid place carpeted with crimson, and crimson-covered chairs and tables, and a pure white ceiling bordered by gold, a shower of glass drops hanging in silver chains from the centre'', evident in the style the features of the house are described: eloquently and with awe and amazement, ``crimson-covered'' being an example of alliteration to emphasise that the entirety of the loving and tender relationship that the colour crimson conveys in stark contrast to the harsh environment of the Heights due to the way the siblings treat Heathcliff. Bront\"{e} makes Heathcliff use sophisticated language, including the metaphor ``shower of glass drops hanging in silver chains'' to convey beauty and, the reader imagines,reflecting light, ``shower''ing. The adjective ``splendid'' at the start of the sentence shows that the quality of the surroundings is totally alien to the pair. It would be, given that the Linton's who inhabit Thrushcross Grange are a lot richer, something that Catherine wishes wholeheartedly for when she returns from recuperating from the dog bite that she received from her escapades in this chapter. 

Overall, both place and setting are heavily explored through different characters' eyes, giving the reader an insight into the social class divides of Victorian England (the Earnshaws compared to the Lintons, for example) as well as the dark and mysterious parts of the Yorkshire moors, allowing the reader to imagine the different lives in great depth.

\end{document}