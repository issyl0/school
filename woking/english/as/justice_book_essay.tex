\documentclass[12pt,a4paper]{article}

\usepackage{parskip}

\begin{document}

\title{How does Capote present the theme of justice on pages 331--333 of In
Cold Blood?  How are the themes of justice and injustice explored in The
True History of the Kelly Gang and elsewhere in In Cold Blood?}
\author{Isabell Long}
\maketitle

In this essay I will attempt to explore the themes of justice and injustice
in several places in both the novels: In Cold Blood by Truman Capote, and
The True History of the Kelly Gang by Peter Carey.  The first is an account
of a brutal murder of a model American family in Kansas in the 1950s and
the ensuing events, and the second is an apparent `true' story of the life
of Ned Kelly, an eventual notorious gang member of Irish origin in
Australia.

In the extract on pages 331--333 of In Cold Blood, justice is a central
theme when Capote describes the hanging of Dick and Perry, the two men that
murdered the Clutter family.  There is a contrast between justice and
injustice when Perry, about to be dropped, whispers ``I apologise'' and
sincerely means it, but is about to be treated in the same way as Dick, the
instigator and murderer who shows a blas\'e attitude towards his own
execution: ``You people are sending me to a better world
than this ever was.'' and asking just before that whether any of the
Clutter family was present, the response followed by Dick seeming
``disappointed'', as if he wanted to go out with a bang, not feeling any
remorse at all, but accepting his execution.

Ned Kelly, the narrator and main character in The True History of the Kelly
Gang, is not terribly highly educated: this is visible to the reader
due to his narration lacking spectacularly consistently in punctuation, and
the use of dialect and taboo words and racism, examples being ``eff'' and
``Chinaman''.  In Cold Blood makes that book much easier to read, due to
Capote's background as a journalist and him wishing to convey the
seriousness and epic description that he could not do if he used a lighter
tone or---as Carey put a lot of effort into doing---stripped punctuation.

Peter Carey explores injustice in The True History of the Kelly Gang on
pages 200--202 when an old man is refused by Ned's mother a glass of brandy
because he could not pay for it and he ``sen[t] a plague among [them]''
which Ned's mother believed that did not disperse until she willingly gave
brandy to the next person who asked, freely.  As a result of this,
she nearly lost her farm and there were ``rats in
the flour and inside the walls and over the bodies of the children'',
forcing Ned's mother to spend some of her little cash on many bottles of
brandy to try to stop the plague.  All this, including the death of several
of her children, was because she did
not give brandy to a rat charmer on the street and was poor and Irish, or
so she had convinced herself.  Ned
himself, the narrator of the story and main character, is shown many an injustice
and commits an injustice when he becomes Harry Power's apprentice and goes
with him to threaten Bill Frost with shooting his ``pizzle'' off to stop
him from cheating on Ned's mother or even liasing with Ned's mother as Ned
does not trust him.  Ned, however, does not see this threat (which does not
materialise) as an injustice and carries on, however he is apprehensive as
he knows killing is wrong: ``I gave Harry the thumbs up but inside my
stomach was churning'', he narrates on page 140, ``churning'' not being a
particularly pleasant sensation, possibly referring also to the unpleasant
thing he was going to try his hand at doing.

Perry's situation is rather similar to Ned's in that he is pushed into a
life that he does not really want, young and impressionably.  Perry hates
killing and wants to be with ``books and maps'' but follows Dick because of
a dream of Mexico and riches; Ned is raised from birth in a poor family on
a poor farm and desperately wants to make something of himself, with
several brothers and sisters who all also vie for their mother's attention.
Ned is also the eldest son and has many responsibilities and a childhood of
turmoil, very like Perry who ran away from home at a young age and ``wets
the bed''.  Like Perry, Ned is dragged into a gang and theiving and
especially killing unwillingly.  Despite this, being quite skilled in
equine care, as Perry who can play guitar and read maps, he wishes to
impress, do people proud and become like his father.  This is shown when
Ned comes home from being mistakenly imprisoned and sees his mother's new
boyfriend who he does not like sitting in his father's chair.  As a result
of this, he ``held the girls'' until Bill ``had to vacate the chair to get
his own potato'', then Ned ``sat down, the moment [Bill] stood''.

\end{document}
