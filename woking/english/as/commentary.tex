\documentclass[11pt,a4paper]{article}

\usepackage{parskip}
\usepackage{fancyhdr}

\setlength{\headheight}{15.2pt}
\pagestyle{fancy}
\lhead{Isabell Long}
\rhead{AS English Language \& Literature coursework}

\begin{document}

\title{Commentary}
\author{Isabell Long}
\maketitle

I have written two literary pieces: the first a dramatic monologue entitled
``Automonobiolography'', and the second a travel writing piece named ``An
excursion to York\ldots''.  My dramatic monologue depicts several life
occurrences of a young woman, drawing from my personal experiences, hence
the name.  The travel writing aims to persuade the reader to visit York, a historic city in the North East of England.  An initial relationship between the texts can be seen from the mention of the train journey in the monologue, and my travel writing: they both reference York.  This relationship helps intertwine the two pieces more due to the fact that both of them talk about journeys, whether they be physical or mental.

I envisaged the first piece as a quite depressing but at the same time
uplifting televised monologue: a woman talking to her reflection about
several life experiences, rather like Joanna Lumley's `Talking Heads'
appearance, only in the glass of a window, not the dressing table mirror.
The second piece is meant to be in a complementary magazine provided by the
train company in their First Class carriages, therefore its audience would
be everyone on the train in First Class, or passing through the carriage on
the way to the Standard Class coaches, picking it up not realising it was a
First Class exclusive.  In order to assure that the travel writing appeals
to and works with `everyone', I have included many different paragraphs
related to many different things, and emboldened several words in each
paragraph related to the group of people I am aiming at with that
paragraph, such as in the sentence ``[i]ts \textbf{nightlife} is
vibrant\ldots''.  I have done this in order to help people skim-reading (as
they usually do when they first see an article) decide which sections are
important to them.  I did not use section headings as, in my opinion, doing
so would have made the text too bitty and regimented.  Both texts are
broken up: the travel writing into nine paragraphs and an image and its
corresponding footnote citing the source of the image, and the monologue
into eight paragraphs and no images.  This splitting is natural in writing
and helps the reader to be able to pause at convenient points and skim read
the first sentence of every paragraph to see, especially in the case of the
travel writing, if he or she would like to read that particular paragraph.
However, reading the texts in bits is discouraged as each text was written
to be read as a whole---especially in the case of the monologue, in order
for the character's story, history and background to emerge.  In
`Automonobiolography', stage directions are used to not only direct the
actor, but to engage the reader in my character's ``shaky laugh[ing]'', for
example, showing the audience that the character is nervous when talking about past
experiences.
 
Short sentences have been used in the monologue to attempt to convey a person's speech, with several hesitations such as ``um'' in the second paragraph, and frequent use of ellipses (\ldots) which indicate trailing thoughts or, in the case of ``um'' too, loss of train of thought as happens incredibly frequently in spoken language, as well as the short sentences (``Right, yes.'' being an example) and onomatopoeia such as ``meh'', which in these modern times is a sound uttered when despairing or not caring.  My monologue character remains anonymous throughout the piece as I think that adds to the suspense and mystery surrounding the character, adding to the reader's feeling sorry for her.

The narrative voice of my monologue is first person, as it would naturally
be as the character is talking about herself.  My travel writing is written
in a mixture of second and third person to help persuade the reader to
travel to York and visit things.  My monologue tells a fictional story of
my character's life, with a few real facts drawn from my life experiences,
similar to the travel writing which has many facts and price figures drawn
from my many trips to York during which I conducted extensive research.
Both of the pieces' sentence types are mostly declarative, with minimal exclamative and interrogative, i.e. rhetorical sentences, as including too many would detract from the travel writing's factual nature, and subtract from the intended serious nature of the monologue.  However, for both, occasional rhetorical questions such as the monologue's ``[it's] important, don't you think?'' help to hold the reader's interest and not render the pieces dry or boring.  In terms of lexis, my monologue character has been well-educated in the south of England, so does not use slang, but wishes to tell things as they are using every day language.  The language in my travel writing is similarly formal, due to it having to appeal to everyone.

In the beginning of the travel writing and throughout the monologue, the reader is reminded of the train setting and, exclusively in the monologue, the cramped environment and uncomfortable seats, by way of mental imagery: ``to reach the restaurant coach on this train I have to [\ldots] squeeze past the bar'', the verb ``squeeze'' emphasising just how tight the spaces are, especially if carrying luggage that in this case the woman did not want to leave in her seat something which every reader (or listener, going by the character's ``I'm talking to myself'' comment) can relate to.  In the travel writing, the reader is warned that if they have kids they should not want to go down the ``steep staircase'' into the ``real'' and ``scary'' haunted house.  An actual photograph is added to the travel writing to make it seem more like the intended magazine article---this image is of ``the iconic, old-fashioned, cobbled, quirky street of the Shambles'', ``iconic'' and ``old-fashioned'' being adjectives describing the street, ``quirky'' being an unusual one that urges the reader to go and visit.

The ending of my monologue aims to leave the reader wondering whether someone did actually come and pick the character up when getting off of her train, in contrast to the travel writing which ends with a semi-rhetorical question asking the reader half seriously, half jokingly, if he or she is going to ``hop on a train [to York]'' too after being so heavily persuaded to visit at the earliest possible opportunity.

\begin{center}
	\subsubsection*{Final word count: 1051 words.}
\end{center}
\end{document}
