\documentclass[12pt,a4paper]{article}

\usepackage{lineno}
\usepackage{parskip}

\begin{document}

	\title{Compare and contrast There Is No God, The Wicked Saith with the Scouting article.}
	\author{Isabell Long}
	\maketitle
	
	\linenumbers
	Text A, There Is No God, The Wicked Saith, is a poem written by Arthur Hugh Clough between 1819 and 1861, about people's differing views about the existence of God, eventually questioning people's motives for believing in God.  Text B is a BBC News article written in April 2012 which deals with the UK Scout movement and a group's claims that it is excluding young people on the grounds of religion.
	
	Text A is comprised of eight stanzas of four lines each, also known as quatrains, every line of each quatrain being of very similar length to the others, indicating well thought out points and a quest to keep the reader engaged.  The article is split into seventeen paragraphs, in this case more often than not one sentence is separated with a blank line, and it is true also that some sentences could be joined to preceeding paragraphs in order to make broader points: ``The NSS [\ldots] says that the Scout promise puts non-believers off joining \slash\ And it asks why[\ldots]''.  This may show that the author John McManus was perhaps trying to make the article one page in length as per his brief, with not very much consideration for literary flow or writing style.
	
	Despite the poem's uniform layout on the page, its rhyme scheme is in fact ABCBDEFE, an irregular one.  Its meter is an iambic tetrameter (i.e. each line has eight syllables), although with catalexis such as ``blessing'' and ``guessing'' on lines two and four, which present further irregularity but a boost to the rhythm, again keeping the reader's interest.  The poem is split clearly into two parts, split with a volta: quatrains one to five are state the viewpoints of the non-believers, whereas quatrains six to eight state the viewpoints of the opposite, more populated group of society, the believers in God.  Clough may have tactically put forward more points against people's belief in God to balance the view of the time when he was writing when most people believed in God.
	
	The poem heavily uses enjambment, shown with the use of commas after the two words quoted in the paragraph above, and indeed in nearly every stanza.  This technique emphasises continuity, clarity, and strong beliefs, yet also at some points allows the reader to absorb a whole sentence in one go and question it, not be stunted by the line breaks, as in the whole of the final stanza when Clough critizes human beings for believing in God only ``when disease or sorrows strike''.  This claim in itself is hyperbolic, as it is most certainly exaggerated, as some people do believe truly in God.
	
	Alliteration (normal, sibilance, and plosive) is used in the poem to punch a few syllables in together to emphasise words.  It is possible that ``shadow'' and ``steeple'' in stanza six could be interpreted as the figureheads in the Church looking down intimidatingly from their high tower of security at the perceived insecure atheists hierarchically below them.
	
	In the title of the poem, Clough labels everyone who say that there is no God as ``wicked'', which is wicked and biased in itself, not accepting, however it is the case that in the 1800s there was not the level of unholiness that we see today, so much of it was seen as malicious as Britain was still very much a Christian country and so it was thought that every respectable citizen was expected believe in God.  This is in complete contrast to the modern age and country we live in now, however, which as we see in the article as embraced multiculturalism.  Curiously, despite being anti-atheist and pro-Christian, in the article we learn that the Scout movement has ``launched a new series of clothing [\ldots] aimed at the growing numbers of Muslim girl[s becoming] members''.  It could be argued that this move is to prevent any racial or religious hatred claims, with the Scout group assuming that a kindly-worded letter from the National Secular Society is a courtesy and not a serious threat to the bad publicity of the movement as atheism is not considered by many to be a religion.  Clough writes about how ``the rich man'', using one of the rarely used adjectives in this poem to emphasise how his views must be important, ``says that [religion] matters very little'' as long as everyone is grateful for their chance in life.  Therefore, religion could be described as a set of moral values intertwined with fiction (``illusion'', in the seventh quatrain) to make people feel more comfortable and brainwash them into doing good on religious grounds, not on moral, common sense and plain decency grounds.  However, whenever something goes wrong or death occurs, people require emotional support and instinctively look for answers in a higher power (``why me?''), whether that power be imagined for that time in their lives or real always in their minds.
		
\end{document}