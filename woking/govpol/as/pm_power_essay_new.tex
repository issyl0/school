\documentclass[a4paper]{article}

\usepackage{parskip}

\begin{document}

\title{Are UK prime ministers as powerful as it is sometimes claimed?}
\author{Isabell Long}
\maketitle

Prime ministerial power depends on many factors including his or her popularity both with the electorate and the opposing party in the House of Commons, the strength or preferable weakness or laziness of the House of Lords, scrutiny on his or her actions, and purely the happening of events.  Due to all of these factors not being good all of the time---with crises such as war scupper organisation and need both parties to come together and work for the good of the country together, ruining parliamentary opposition, for example---prime ministerial power levels have been variable over the years.

The powers of the House of Lords are important when considering the lack of power of a prime minister in the United Kingdom.  The House of Lords exists to scrutinise what government does, by way of committees and reports and regular meetings.  However, the current prime minister can exert some control over the House because he has the power of patronage of the Lords: he can appoint whoever he wants to the Cabinet and the Lords in order to ensure that his party has enough of a majority in the House who will vote with their party, in order to ensure that bills are passed in the governing party's favour, and if that doesn't happen, the Prime Minister can threaten the use of the Parliment Act, deferring the passing of a Bill for a year, but then automatically passing it without further debate.  This again strengthens his power.

Prime ministerial power has increased in recent years through the breakdown of the official principle of Royal Prerogative, the principle that the Queen had lots of power and could only appoint Lords and declare war, things amongst many others which the prime ministers have now taken control of: some now see the Queen and the Royal Family as traditional, touristic figureheads, as they have so little power due to modern developments in the workings of government, despite the Queen meeting with the PM and keeping herself in the loop as to goings on, and having to officially accept and sign the government into power.  It is still said that the Queen must sign every Act of Parliament that is passed, also known as giving Royal Ascent, but according to the website of the UK Parliament, this has not happened since 1854: now she implicitly agrees with every one.

Scrutiny can lead to prime ministerial power decreasing, due to parliamentary opposition, being unfavourable to the electorate, and the country's position in the world.  Prime Ministers' Questions used to be a good way for the opposing party to scrutinise the work of government by asking pointed questions, but in recent years it has gone downhill and many people consider it to be a waste of half an hour to watch the leader of the government and the leader of the opposition cutting into each other with personal, childish remarks and much noise and heckling, always blaming the other, not taking much responsibility for their actions.  Scrutiny can be good, however: Parliamentary and Departmental Select Committees are quite effective in scrutinising not just the Prime Minister but various departments, and if the findings are not good, the Prime Minister can sack a minister, or the minister can resign for having broken rules or---if in Cabinet---Collective Cabinet Responsibility, increasing prime ministerial influence and power.

Parliamentary sovereignty and the UK's position in the world generally and economically, including in media coverage, is a hindrance to the Prime Minister as it is so variable.  With the introduction of devolution in Wales, Scotland and Northern Ireland, a lot of power has shifted to their respective Parliaments or Assemblies and is no longer in Westminster which means that is is potentially harder for the Westminster Prime Minister to keep the country together and easier for him to go out of favour.  The EU does not help this: even though the UK can withdraw from the EU at any time, a lot of our sovereignty has been devolved to them, especially in the case of the Human Rights Act.  Now, there are moves afoot to scrap the Human Rights Act and introduce a British Bill of Rights, similar to that of the United States, to give us more power.  This would be useful for us if ever another Abu Qatada came into the country and we wanted to deport him quickly despite him not having been charged with anything or put on trial, without going through the European Court of Human Rights because of the Human Rights Act: Theresa May might be happy.  Also, a Bill of Rights would give citizens more of an idea of their rights in a definitive document: due to the UK's unwritten constitution, we as citizens have very little idea of what we are allowed within the law compared to US citizens, for example.  David Cameron is in a difficult place with many Eurosceptic backbench Conservative MPs revolting regularly against the UK's membership of the EU.

Favourable prime ministers have to be charismatic, have parliamentary majority, and inherit the job in a reasonable economic climate: John Major faced an incredibly low parliamentary majority and struggled.  Gordon Brown also got hounded after Tony Blair left government as he was `unelected', despite the fact that only MPs are elected by the people: the PM is the leader of the party he belongs to, and chosen internally.  Margaret Thatcher and Tony Blair were very influential and changed lots---some better, some worse.  War and events can make or break a PM's career: some may say that the Falkland Island victory for Thatcher made her popular, whereas in the case of Blair, Iraq could have been said to do the opposite.

Overall, prime ministerial power is incredibly variable and depends on many factors: it is up to the prime minister of the time to make the most of his or her situation and convince people that he or she is the right person for the job, gaining support and then being able to exert authority and portray himself or herself confidently to at least give the illusion of being powerful if it happens that backbench revolts, sovereignty, unfavourable opposition, events such as war, and his or her majority lead to him or her having less power than required.

\end{document}