\documentclass[a4paper,12pt]{article}

\usepackage{parskip}

\begin{document}

Political parties \textbf{do} promote the democratic process by:

\begin{itemize}
	\item{representativeness---multiple views}
	\item{a group of officials (govt) can be held accountable because they believe in the same core values and stick together}
	\item{they can campaign around election time}
	\item{people vote for parties, so direct their frustration at the party as a whole, not just at one person}
	\item{party speeches around elections---groups, not just individuals}
	\item{group policies and manifestos}
	\item{political training; organised hierarchy; administration for potential promotion}
	\item{vote of no confidence---legitimacy}
	\item{effective decision making}
	\item{party discipline---people within the party who do wrong can be kicked out or shunned if the party so chooses}
\end{itemize}

Political parties \textbf{do not} promote the democratic process by:

\begin{itemize}
	\item{the main three parties are very similar}
	\item{contemporary politics involves appealing to the middle ground, the masses; party politics is too consensus-based}
	\item{party voting limits constituency representation---people vote for parties in their heads, not MPs}
	\item{party makes complex ideas into one amorphous (mud pie [of a]) whole}
	\item{parties limit individual expression---party line}
	\item{representativeness issues}
	\item{party leaders not necessarily representative of the grass roots of the party}
	\item{we do not get a say over who is the party leader, just over which party gets elected and governs}
	\item{power of PM and Executive makes it difficult for us to scrutinize internal party workings}
\end{itemize}

\end{document}
