\documentclass[12pt,a4paper]{article}

\usepackage{parskip}
\usepackage{fancyhdr}

\pagestyle{fancy}
\lhead{Isabell Long}
\rhead{25th March 2012}

\begin{document}

\title{Assess the main factors which limit the effectiveness of Parliament.}
\author{Isabell Long}
\maketitle

Parliament performs many functions: scrutinisation of legislation and the government through Ministers themselves and\slash or Select Committees, passing of legislation, and representing the citizens of the UK that they are in the House of Commons thanks to.  These reasons can lead to limited success and effectiveness, however, through ineffective opposition to the government from the opposing party, sovereignty, the existence of the House of Lords, and backbench MPs revolting, to name a few.

Collective Cabinet Responsibility is the principle that, in the Cabinet, if the decision made is positive but members of the Cabinet do not agree with it personally, they have to be seen to agree in public as they made that decision and take that responsbility to not speak ill of the decisions taken.  If they do not, the member of the Cabinet must resign.  This limits the effectiveness of Parliament in that speech is not totally free: politicians cannot speak their mind, they have to toe the line.  This ties in also with the whips outside of the Cabinet, in the main House: they keep Party members in line and force them to vote with their Party, in some cases, leading to a breakdown of true freedom of vote and less scrutiny when it comes to legislation as everyone in the Party putting forward the piece of legislation has to not scrutinise it, or, if they do, go against their views if they object and vote for a piece of legislation.

A large obstacle to the functioning of Parliament and the government's law passing is the House of Lords.  Take the recent Welfare Bill as an example: it went backwards and forwards, being scrutinised, and was eventually passed.  Some would argue that there is no point to the House of Lords, because of this toing and froing and the fact that Parliament does not have to take into account the Lords' recommendations is a waste of time and money.  Nick Clegg has proposed reforms in order to stop it being so much of an obstacle to Parliament getting what it wants, and to make it more democratic.

In conclusion, Parliament plays a very important role, but sometimes its processes of scrutiny can lead to it not getting as much done as it could.  It being effective depends on the mood of the country which may or may not make law passing easier, the Prime Minister's leadership, the amount of participation from MPs, and the number of MPs from each political party who may or may not hinder progress.

\end{document}