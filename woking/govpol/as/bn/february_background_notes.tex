\documentclass[12pt]{article}

\usepackage{parskip}
\usepackage{fullpage}

\begin{document}

\section*{Unit 2, section 2.2 (Parliament) background notes}

\subsection*{Notes from textbook}

Parliament is known as the legislature, however its main role is that of scrutinising the government and legislature.

Different types of governments exist: presidential and parliamentary.  The UK has a parliamentary government, due to the UK Parliament having sovereignty to exercise political power (and the EU, but we can pull out at any time, so this is not a great problem).  The government is drawn from the two houses, so there is a fusion of powers between the legislature and the executive.  The government must also be accountable to Parliament.

Presidential government: Legislature and executive separately elected; the president $\neq$ part of the legislature; president directly accountable to the people; separation of powers between the executive and legislature; codified constitution that separates them.

Select Committees and general committees exist to scrutinise on a particular issue, or group of issues.  Each government department has a committee.  As of 2010, there were 19 of these committees.  These exist for the government departments in departmental question time, for example, scrutinising their workings.  The House of Lords has no equivalent, just general committees to scrutinise legislation made in the Commons and general goings on.  Cabinet Committees exist in many different areas of policy and life, and the civil service plays an important role in advising due to its neutrality.

An example: in 2006, the Home Affairs committee produced a report rejecting the 90 day detention without trial case, and the report was accepted by the House of Commons.  Committees consult and report on issues, and deliver reports to the House of Commons\slash Lords, allowing them to take them into account or not.  Usually they do.

Both houses, the HoL and the HoC, have a Speaker.  The Speaker's job is to oversee the debates, select speakers from the floor, and arrange the House's business.  The current House of Commons Speaker is John Bercow.

The House of Lords, and its reform, is contentious.  Many have wanted to reform the HoL for years.  Nick Clegg outlined in 2010 several ways in which the Coalition plans to do this: a partly elected chamber, etc.  The HoL is not currently elected, and so not accountable, and they scrutinise legislation put forward by the HoC.  Their function is an important one.  Though they could be said to be out of touch with reality, being Lords.  Now, the Prime Minister can appoint whoever he wants to the House of Lords.

\subsection*{Definitions}

\textbf{Parliamentary government}: A system of politics where government is rawn from Parliament and is accountable to Parliament.  In other words, the government has no separate authority from that of Parliament.

\textbf{Westminster model}: A democratic parliamentary system of government modelled after the politics of the United Kingdom.

\textbf{Representative and responsible government}: A government that is representative is one that represents the view people it was elected by; a responsible one is one that takes care and acts responsibly and looks after the interests of the country and helps.

\textbf{Presidential government}: A president normally has a separate source of authority from that of the legislature.  This means that the executive (president) is accountable to the people directly, not to the legislature.

\textbf{Fusion/Separation of powers}: The fusion of powers is the opposite of the separation of powers.  The latter is a constitutional principle that the three branches of government---the legislature, executive and judiciary---should have separate membership and separate powers and should be able to control each other's powers, which is largely absent in the UK.

\textbf{Bicameralism}: Parliament is composed of two legislative bodies, two chambers: the House of Commons and the House of Lords.

\textbf{Accountability}: Parliament is accountable to the electorate because it is made up of MPs that were elected in a general election as leader of their constituency, representing a particular political party.

\subsection*{Recommendations of the Wright Committee in 2010}

\begin{itemize}
	\item{Reduction in the size of committees to a standard number of members.}
	\item{E-petitions system---this was introduced.}
	\item{One backbench motion per month should be routinely scheduled for debate.}
	\item{Members of departmental committees and other committees should be elected from within party groups by secret ballot.}
\end{itemize}

The Coalition agreed to bring these recommendations, and the Committee's other recommendations, into full practice.

\end{document}
