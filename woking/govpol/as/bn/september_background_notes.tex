\documentclass[12pt]{article}

\usepackage{fullpage}
\usepackage{parskip}

\begin{document}

\section*{Unit 1, section 1.1 (democracy and political participation) background notes}

	\subsection*{Notes from textbook}
	
		Rousseau said that the English think they are free, but they are only free once every five years at a General Election.

		Abraham Lincoln said that:
		
		\textsl{Democracy of the people} -- opportunity to participate.  If citizens $\not=$ active then democracy withers and dies.\\
		\textsl{Government by the people} -- people themselves make decisions regarding what affects them $=$ direct democracy \textsc{or} intensely sensitive to public opinion.\\
		\textsl{Government for the people} -- whoever governs does so with the broad interests of citizens in their head, not their own interests or only one section of a community.
		
		\textbf{Democracy's importance}
		
		Democracy establishes and protects freedom.  It ensures that no government can threaten freedom without the express consent of the people.

		Democracy protects minorities' existences and interests, and ensures that they have a free voice and are free from discrimination.

		Democracy controls government power because a government can be held to account, re-elections can be called, and governments should be controlled by elected representatives of the people.  This way, people feel safe from corruption.

		Democracy encourages political participation because the people feel more like they can get involved and will be listened to.  Democracy is critical in assuring the prevention of tyranny (harsh ruling).
		
		Democracy disperses power more widely, i.e. no small elites but big governments, civil service, people power, etc.
		
		\textit{Democracy has its flaws}: it may not always be the most appropriate political system.  In religious countries, for example.
		
		A citizen of a country enjoys the rights and privileges of that country.
		
		Under \textit{absolute monarchies}, the people of a country were expected to perform their duties, obey and not question the legitimacy of government.
		
		\textit{Conscription (joining armed forces)} could be imposed on a citizen according to his or her obligations as a citizen of the country---he or she will not have complete freedom, but will have rights.
		
		Pressure groups are growing in popularity (textbook page 31).
		
		AV referendum (UK) to change voting system -- result: \textsc{no}.
		
		Referendums: most direct form of democracy.  People express their views and consent (or not) to a change.
		
		Problems with referendums: issues too complex for people to understand; campaigns are expensive; people might use referendums to express dissatisfaction with government and ignore the actual issue---emotional rather than rational; tyranny of the majority.
		
		Representation (gender, ethnic minority, university educated) politically in the UK (table page 44).
		\begin{itemize}
			\item{Women: 22\% in the HoC; 51\% whole population.}
			\item{Ethnic minority: 4\% in HoC; 8\% whole population.}
			\item{University educated: 90\% in HoC; 31\% whole population.}
		\end{itemize}
		
		Representative democracy: elected representatives expected to use superior knoowledge and judgement to consider issues deeply.
		
		Britain is a tolerant society (example: immigration, benefits).
		
				
		
		
	\subsection*{Definitions}

		\textbf{Democracy}: Governments should serve the interests of the people living and elegible to vote in that country.  A principle of democracy is that the people have the right to vote.

		\textbf{Legitimacy}: When a government's right to govern is accepted.

		\textbf{Representation}: People elect representatives to make decisions on their behalf (not delegates).  Example: MPs in constituencies. 

		\textbf{Direct democracy}: Every citizen has the right to have a say in decision making.  This was the first form of democracy; it came from Athens.

		\textbf{Representative democracy}: Everyone can choose who has the right to have a say in decision making.  The people are represented, but in Britain this is done by the MP voting in what he or she believes to be the best interests of the constituents in her constituency, not necessarily what they tell him/her to vote.

		\textbf{Liberal democracy}: One person has one vote so that elections are free and fair.  Multiple parties exist, so they can supposedly represent everyone---this is pluralism.  The Civil Service is impartial and advises whatever Party is in power on how to best implement its policies.  The press and Pressure Groups are free to publish what they like.

		\textbf{Parliamentary democracy}: Different from a Presidential democracy, i.e. one where there is a President.  In a Parliamentary democracy, for example this country, we have an unelected Head of State (the Queen), an unelected Prime Minister (as we vote for the party not the person in charge of the Party, in theory), and the Prime Minister elects his Cabinet without consulting the people of his country.  In Government, the Lords are not elected either.  However, the Government is still accountable, legitimate, scrutinised (by the opposition) and so can pass decent laws, and the UK is subject to Sovereignty, the idea that no body can override legislation made by the current Government---apart from if it contravenes EU law.

		\textbf{Political participation}: Opportunities for people to get involved in politics.  Examples: voting; debating; watching the news; standing as a candidate at local/national level; joining a party etc.

		\textbf{Referendum}: A referendum is a vote where the people are asked to vote `yes' or `no' to a particular policy change.

	\subsection*{Notes on the \textsl{`How democratic is the UK?'} article from Volume Two of November 2010's `Politics Review' magazine}
	
	Democracy means `rule by the people' or `people power', from the Greek word `demos'.
	
	Edmund Burke said ``your representative owes you not his industry only but his judgement \ldots and he betrays you if he sacrifices it to your opinion'' in relation to representative democracy.  The representatives are not just delegates sent to vote on Parliamentary issues with a specific vote to cast, determined directly by the people.  In a representative democracy, citizens elect representatives who enact laws on their behalf.
	
	In Athens, the purest form of democracy was alive: direct democracy.  This meant that citizens had direct input into the decision making process.  Direct democracy is still used, albeit rarely, today: it comes in the form of referendums where the people get the chance to vote on an issue more than just who becomes the government once every five years.
	
	In the UK, democracy is characterised by free and fair votes with secret ballots, human rights (the right to vote being crucial), and the existence of a wide range of political parties and pressure groups to oppose and contest (a pluralist democracy).
	
	The UK does not have a written Constitution like the United States does, and this is a flaw: where can citizens go to get an official `low down' on all the laws and their rights?  Privatisation is a problem, as the citizens did not elect certain bodies to run certain aspects of their country, yet the government farms out certain roles to these private companies.  First-Past-The-Post, the UK's electoral system, is not very fair either: is is not proportional and there is no counting of people who come second.  This is important when you consider that no party has received over 50\% of the vote at a General Election for a large number of years.  Dissatisfaction with the country's leadership has given rise to a number of extreme groups; even though these remain in the minority as far as support is concerned, they still exist, although some would say this adds to the diversity and is in fact what democracy is there for: to be pluralist.
	
	The numbers of people voting in General Elections have decreased over recent years: 84\% in 1950 compared with 62\% in 2001.\\
	There has been an erosion of civil liberties since the attacks in the USA on 9/11 (2001) and London on 7/7 (2005).\\
	Trust in politicians has declined in the wake of a series of scandals (example: expenses).
	
	\textsc{But}:\\
	Human Rights Act passed in 1998: set rules for equality and rights for citizens.\\
	Freedom of Information Act passed in 2000: citizens could legally file a request for a certain piece of information if it was not publically available but they had a good reaosn to view.\\
	Devolution has meant that the Welsh, Scottish and Northern Irish are recognised as their own countries, not just as part of the UK.
	
	A \textbf{recall} is mainly used in the United States.  It allows citizens to initiate a referendum to remove their elected representative before the end of their term in office.  It is normally only permitted where there is evidence of corruption or incompetence.
	
	A \textbf{citizens' jury} is a panel of citizens that hears evidence and delivers their verdict on a government proposal.  Moslty used in Germany and the United States, but encouraged by Gordon Brown here in the UK.  These juries do not have the final say on government policy.
	
	A \textbf{citizens' assembly} is similar to a citizens' jury, but involves more people (hundreds or thousands as opposed to tens) who are given more power to decide upon a course of action.
	
	UK democracy is \textsl{evolutionary} rather than \textsl{revolutionary}: all three main UK parties appear committed to rebalancing democracy in favour of `the people', by giving citizens a greater input into the decisions that affect them.
	
\end{document}
