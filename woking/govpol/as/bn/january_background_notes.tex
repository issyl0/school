\documentclass[12pt]{article}

\usepackage{parskip}
\usepackage{fullpage}

\begin{document}

\section*{Unit 2, section 2.1 (the UK constitution) background notes}

\subsection*{Notes from textbook}

Constitutions can be codified and uncodified.  Most countries have a constitution of some kind, but not all of them are codified.  The United Kingdom does not have a codified constitution, for example.

Constitutions are useful in that they determine how political power should be used within a state, they establish the political processes that make the system work, they state what the limits of governmental power should be, assert the rights of citizens against the state, and establish nationality rules.  Constitutions can be ammended from time to time---they are not totally fixed forever---otherwise after centuries, they would be very old and no longer serve the needs of the country, because the country would have evolved.

The British constitution is not codified in full, but there are parts of it, such as the Human Rights Act that comes from the European Union and that Britain has agreed to, that are.

Codified constitutions are generally entrenched, so ammendments cannot be made without a long, drawn out process passing through several bodies of power before being finally agreed, whereas uncodified ones are much more flexible.

Judicial review is the process that the judges and lawyers have to go through of reinterpreting law, because the uncodified constitutions change so much and there is nothing concrete for them to refer to.

Arguments for and against a codified constitution:

For:
\begin{itemize}
	\item{Judicial review would be more transparent and easier}
	\item{Citizens would be more aware of what their rights were if it was all written down and easy for them to refer to}
	\item{The UK would line up with most other modern democracies, and its relationship with the EU would be better}
	\item{It might prevent further drift towards excessive executive power}
	\item{Better safeguarding of citizens' rights}
\end{itemize}

Against:
\begin{itemize}
	\item{Executive government is strong and decisive because of the lack of constitutional constraints}
	\item{The UK's conventions that have descended through the years would be difficult to write them down}
	\item{``If it ain't broke, don't fix it''}
\end{itemize}

The main sources of the UK constitution are:

\begin{itemize}
	\item{Statute law}
	\item{Conventions}
	\item{Historical principles}
	\item{Common law}
	\item{Tradition}
\end{itemize}

The prerogative powers of the monarchy have eroded---no King or Queen has signed a bill for Royal Assent since Queen Victoria, even though they can technically.  Now, however, the monarchy exists more as an attraction---Parliament has the most power signed over to it from the monarchy.

The UK is a unitary political system.  This means that Parliament has the ultimate say in political matters, despite devolution and EU membership.

The UK's lack of codified constitution means that the UK Parliament can act decisively and quickly without a long, drawn-out process.  Having a codified constitution would possibly increase political participation due to the public realising their rights and having more confidence in the government following written, relatively inflexible, constitutional rules.

Legal sovereignty is the ultimate power to make laws in the state.

Political sovereignty is where political power \textit{really} lies.

Sovereignty: EU law is superior to UK law.  If a UK law contravenes an EU law, the EU can ask for the UK to repeal it.  The UK does, however, have the option to leave the EU at any time to avoid having to do this, and regain total sovereignty.

An proposal of constitutional reform nowadays is the axing of the House of Lords, replacing it with a fully elected second chamber.

\subsection*{Definitions}

\textbf{Constitution}: A set of principles, which may be written or
unwritten, that establishes the distribution of power within a political
system, relationships between political institutions, the limits of
government jurisdiction, the rights of citizens and the method of amending
the constitution itself.

\textbf{Constitutionalism}: The idea that governments can and should be legally limited in their powers.

\textbf{Codified\slash uncodified constitution}: A codified constution is one that is written, whereas an uncodified constitution is more flexible due to it being unwritten generally in full.  The UK constitution is uncodified.

\textbf{Devolved\slash federal government}: Devolved government is where powers are granted to regions of countries but can be revoked, such as the Scottish\slash Welsh \slash NI powers.  Federal government is where powers are constitutionally assigned to regions: such as each US state has the power to impose the death penalty, and the main government cannot do anything about whether they do or not.

\textbf{Parliamentary sovereignty}: The British principle that Parliament
is the source of all political power (with the exception of all prerogative
powers of the prime minister) and that enforceable laws must be approved by
Parliament.

\textbf{Pooled sovereignty}: The sharing of decision making powers between states.

\textbf{Devolution}: A process whereby power, but not sovereignty, is
transferred to regions and countries.  It means the establishment of
assemblies and governments in those regions.

\textbf{Quasi-federation}: A division of powers between central and regional government that has some features of federalism without possessing a formal federal structure.

\textbf{Elective dictatorship}: When, due to weakness in Parliament, the government is able to behave like a dictator.

\subsection*{Key constitutional reforms since 1997}

\begin{itemize}
	\item{Devolution of some power to the Scottish Parliament}
	\item{Devolution of some power to the Welsh Assembly}
	\item{Devolution of some power to the Northern Ireland Assembly}
	\item{The Human Rights Act 1998}
	\item{Constitutional Reform Act 2005}
\end{itemize}

\end{document}