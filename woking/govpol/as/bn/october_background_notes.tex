\documentclass[12pt]{article}

\usepackage{parskip}
\usepackage{fullpage}

\begin{document}

\section*{Unit 1, section 1.2 (party policies and ideas) background notes}

	\subsection*{Notes from textbook}

	Political parties must be organised in order to carry out their many functions.

	Parties must develop policies and programmes to present to voters in order to secure election of their candidates, and gain public support for them.

	The Green Party has always been reluctant to recognise leaders, as it is a democratic party that is highly suspicious f the exercise of political power.  Though, parties need successful leaders to be successful and have some direction---hierarchy is a must.

	Functions of political parties: see main notes.

	Parties are usually described as right or left wing.

	Traditional conservatism:

	\begin{itemize}	
		\item{Roman Catholic view that man is born with original sin, so is ultimately flawed}
		\item{People on the whole are untrustworthy and self-seeking, so mankind is in need of a firm and strong government}
		\item{Edmund Burke: ``the relationship between government and the people should be similar to that between a parent and child''}
		\item{Human beings naturally think of themselves first}
		\item{Causes of crime and disorder lie with the individual}
		\item{Conservatives prefer strong governments and favour the needs of the community}
		\item{An absolute ruler needs to govern to protect the people from themselves}
		\item{Conserve what is seen to be good, and reform what is proving to be undesirable}
		\item{Pragmatist: flexible approach}
		\item{Empiricist}
		\item{Home owners are seen as good and having a greater interest in ensuring that society is stable}
		\item{Equality of outcome}
	\end{itemize}

	New Right conservatism:

	\begin{itemize}
		\item{Originated in the late 1970s}
		\item{Deregulation: privatisation of industry and introduction of more competition}
		\item{Disengagement: reluctance to interfere in the economy or support failing industry (\textsl{laissez-faire})}

		\item{Opposition to trade unions}
		\item{Low levels of personal\slash business tax to incentivise wealth}
		\item{Opposition to people being dependent on benefits}
	\end{itemize}

	David Cameron's conservatism:

	\begin{itemize}
		\item{Society is responsible for improving the conditions of deprived people by opening up opportunities for them}
		\item{Claiming benefits should be more difficult}
		\item{Taxes must be set at a level that incentivises work}
		\item{Environmental protection is the highest priority and everyone's responsibility}
		\item{Tough approach to serious crime, but a balanced approach to understand why crimes are committed}
		\item{The state must play a key role in reducing poverty and increasing opportunities, and citizens should have more local control over the state}
		\item{Education should not single out the more fortunate}
		\item{Basic rights and liberties should not be sacrificed for the sake of security}
		\item{HoL\slash HoC reform $+$ AV referendum}
	\end{itemize}

	Tony Blair's `Third Way' Labour Party (not the same as original socialist principles (see pages 73--80 of textbook)):

	\begin{itemize}
		\item{Capitalism should be allowed to flourish and the state should enforce competition and fair trade}
		\item{Limited role for trade unions: individual workers' rights protected by law}
		\item{Benefits used to incentivise work}
		\item{Preservation of high-quality public services}
		\item{Public sector borrowing $=$ acceptable if used for investment in public services}
		\item{Tough on crime, tough on causes of crime}
		\item{Britain must have a leading role in world affairs and remain at the centre of the EU, albeit with its own currency}
		\item{Reform, but not a radical one: decentralisation of government, minor Lords reform, and the Human Rights Act}
	\end{itemize}

	Liberal Democrat values:

	\begin{itemize}
		\item{Equality of opportunity (along with Labour)}
		\item{Fair taxes}
		\item{Keep the welfare state}
		\item{Tackle the causes of crime by education}
		\item{Constitutional reform: stronger Human Rights Act, more power to Scotland\slash Wales\slash Northern Ireland\ldots}
		\item{Stay in the EU}
		\item{The environment is important and the government should be more committed}
	\end{itemize}

	Other parties:

	\begin{itemize}
		\item{Scottish National Party (SNP)}
		\begin{itemize}
			\item{Scottish independence}
			\item{Other policies similar to those of the Liberal Democrats}
		\end{itemize}
		\item{Plaid Cymru (PC), the Welsh Nationalists}
		\begin{itemize}
			\item{Does not demand full independence for Wales}
			\item{Committed to the development of a completely bilingual Wales}
			\item{Other policies socialist-leaning}
			\item{Supportive of environmental policies, especially countryside and agriculture}
		\end{itemize}
		\item{UK Independence Party (UKIP)}
		\begin{itemize}
			\item{Breakaway from the Conservative Party}
			\item{Britain should withdraw from the EU}
			\item{Nationalist}
		\end{itemize}
		\item{Green Party}
		\begin{itemize}
			\item{Outside of environmental policy, most other policies from the Liberal Democrats}
			\item{However, greater emphasis on tolerance and human rights}
			\item{Great concern over environmental issues and animal rights\slash biodiversity}
		\end{itemize}
	\end{itemize}
	
	\subsection*{Definitions}

	\textbf{Political party}: An association of people who have similar political philosophies and beliefs.  Normally, a party will seek power and develop an organisation whose purpose is to fight elections.

	\textbf{Left/Right}: The two sides of the political spectrum.  See textbook pages 55--57 for the main differences.

	\textbf{Liberalism}: A state of political mind or a political movement that places freedom, rights and tolerance high on its scale of values.

	\textbf{Conservatism}: A state of mind and a political movement that is naturaly averse to excessive change and reform.  It is sceptical about strongly held political views, prefers the known to the unknown, and generally supports the retention of traditional institutions and values.

	\textbf{Socialism}: A state of mind and a political movement that places such values as equality of opportunity, social justice, and collectivism high on its scale of values.  It is either opposed to free market capitalism or proposes measures to moderate the undesirable effects of capitalism.

	\textbf{Factionalism}: A group of people forming a usually contentious minority within an already established group. 

	\textbf{Consensus politics}: A circumstance where two or more major political parties broadly agree on most basic policies.  In other words, a period when there are few or no major political conflicts.  It may also refer to a single issue where differnt parties agree to support the same policies. 

	\textbf{Adversary politics}: The opposite of consensus.  This is is a circumstance where political parties are engaged in consierable conflict over political issues.

	\subsection*{Differences between Green Party and Conservative polices on environment}

	Conservative:
	\begin{itemize}
		\item{Environment}
		\begin{itemize}
			\item{Encourage home efficiency via the Green Deal}
			\item{Optimise energy from renewable sources}
			\item{New nuclear power stations}
			\item{Roll out smart meters}
		\end{itemize}
	\end{itemize}

	Green Party:
	\begin{itemize}
		\item{Environment}
		\begin{itemize}
			\item{Phase out nuclear energy}
			\item{Remove, Reduce, Replace}
			\item{Introduce smart meters}
			\item{Introduce stronger planning policies to support onshore wind, tidal, wave, solar and geothermal energy schemes}
		\end{itemize}
	\end{itemize}

	\subsection*{Notes on the Coalition's plan for constitutional reform, Politics Review Volume Three, February 2010}

	\begin{itemize}
		\item{A smaller, cheaper House of Commons}
		\item{Petitions to Parliament (e-Petitions?)}
		\item{US-style `recall' elections for `misbehaving' MPs}
		\item{AV referendum}
		\item{Greater powers for the Welsh assembly}
		\item{House of Lords reform}
		\item{British Bill of Rights}
		\item{Freedom Bill}
		\item{Tighter regulation of parties' finances}
	\end{itemize}

\end{document}
