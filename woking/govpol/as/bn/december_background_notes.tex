\documentclass[12pt]{article}

\usepackage{fullpage}
\usepackage{parskip}

\begin{document}

\section*{Unit 1, section 1.4 (pressure groups) background notes}

\subsection*{Notes from textbook}

Pressure groups are organisations that seek to influence public policy.
Pressure groups influence decision making---they are not part of
government, so do not make decisions themselves.  Pressure groups educate,
informing the public about issues that may concern them and therefore
helping them to develop their own views.

Pressure groups represent their followers in government, and political
participation is easier for the public because pressure groups focus on a
narrower range of issues, so more people may agree with them.

Pressure groups are different to parties in that they:

\begin{itemize}
	\item{do not seek power}
	\item{concern themselves with a narrow range of issues}
	\item{are not and do not have to be accountable for their actions}
	\item{sometimes act illegally or promote civil disobedience}
\end{itemize}

Insider pressure groups have found a way into government and can influence
MPs and ministers and policy directly through having a direct say in
policy-making.  This is good when many people may be disgruntled by
something government does: government can consult the pressure group.  If
the pressure group was not inside, the government would not necessarily
know the views of the public on a select, possibly local or localised
issue such as farming or the banning of fox hunting.  Outsider pressure
groups still just have to influence and mobilise public opinion and then
hope that they get enough of a following for government to notice them.

New social movements, forms of pressure groups, are often very fast-moving
and grow rapidly, however they are prone to disappearing or merging with
other, similar organisations, when they become popular and they see their
task as done.  An example of this is `Make Poverty History' which had great
success in 2005 but now seems to have disappeared.

Some pressure groups include:

\begin{itemize}
	\item{Greenpeace}
	\item{Shelter}
	\item{British Film Council}
	\item{Friends of the Earth}
	\item{Countryside Alliance}
\end{itemize}

Smaller and larger groups exist.  Larger groups include Age UK, Friends of
the Earth, and the Countryside Alliance.  Pressure groups tend to be less
organised than political parties, and so more easily accessible to people
as there is less heirarchy, however there are defined leaders and,
particularly within insider pressure groups, those leaders may not listen
to their following but push for themselves and their own, personal agendas,
not those of the minority.

Pressure groups have increased in importance over the years, due to the
public's disillusionment with the political parties, the narrowing of the
political spectrum so the people being disgruntled about not having their
select, possibly extreme views catered for, so to speak, or political
parties' policies covering too wide and too general an area.  Also, people
find that joining a pressure group is better than joining a political party
because fellow members don't believe in one or two things broadly related
to, say, what party $x$ stands for, but specifically (but not necessarily
purely) the one specific thing that pressure group $y$ stands for and
campaigns for or against.

It is important to note that terrorist groups are \textbf{not} classed as
pressure groups.  Pressure groups should not be violent, however they can
take what is known as direct action: protesting, lobbying, etc.

Pressure groups can succeed or fail, like all groups.  Some reasons for
success or failure may include:

\begin{itemize}
	\item{current political climate}
	\item{election proximity}
	\item{celebrity support}
	\item{opposition strength}
	\item{membership, money, and therefore resources}
	\item{organisation}
\end{itemize}

Relating to the celebrity support mentioned in the list just above, Jamie
Oliver campaigned not long ago for healthier school meals, which caused
much controversy but also gained a big following.

Disadvantages of pressure groups within a democracy are that they are
always in the way, they detract from main issues onto smaller issues that
not everyone may care about, and they may have too much say or be too
forceful.  They are, however, advantageous in that they give a voice to the
underrepresented, sometimes minority views, and/or actually do things
instead of just talk about doing them.

\subsection*{Definitions}

\textbf{Pressure group}: An association that may be formal or informal,
			 whose purpose is to further the interests of a
			 specific section of society or to promote a
			 particular cause by infliencing government, 
			 the public, or both.

\textbf{Sectional/promotional groups}: Sectional pressure groups represent
				       a section of society such as a trade
				       union, whereas promotional presure
				       groups seek to promote causes. 

\textbf{Insider/outsider groups}: Insider pressure groups operate inside
				  the political system and have contact
				  with ministers and official committees
				  and are regularly consulted by
				  government.  Outsider pressure groups
				  seek to influence by convincing the
				  public.

\textbf{Pluralism}: Many different groups are allowed to exist and operate.

\textbf{Elitism}: Power monopolised by small groups of influential people. 

\subsection*{Information about Friends of the Earth from their website}

Friends of the Earth are an environmental organisation who take action and
campaign against climate change and other environmental issues such as food.

Their three main ideas are that:

\begin{enumerate}
	\item{everyone deserves a good life and have a fair share}
	\item{the people and the environment need to work with the
	      government, not against each other, and need to accept}
	\item{people need to live within the limits of the natural world so
	      that their children and children's children still have a
	      planet to live on}
\end{enumerate}

Recently, Friends of the Earth have mounted (and won) a legal challenge
regarding the recent solar panel Feed-in-Tariff reductions which claimed
that the cut-off date---so soon after the announcement and before the end
of the governmental consultation period---was unlawful, so they are acting
in the interests of all the businesses and people affected by the downfall
(due to this) of the (previously thriving) solar panel industry and its
customers.

\end{document}
