\documentclass[12pt]{article}

\usepackage{parskip}
\usepackage{fullpage}

\begin{document}

\section*{Unit 1, section 1.3 (elections) background notes}

	\subsection*{Notes from textbook}

First Past the Post (FPTP) is the electoral system used in Britain for
elections.  Voters get one vote, and they select one candidate to represent
them in their constituency.   The person with the most votes wins the seat
in the constutiency, and so the party with the most overall seats wins the
election and gets to form a government.  FPTP is not proportional, and
smaller parties find it difficult to `break in' to the ranks of the parties
and gain votes.  Votes are not of equal value---some don't count, in the
case of where, say, a Conservative MP is in a `safe seat'---no votes, even
tactical ones, will make any difference.  Constituencies are the same, and
so there are strong constitutional links, with every constituency having an
MP, so people feel represented.

First Past the Post is a \textbf{majority} system.  An overall majority,
(possibly 50\% of the vote) is needed to get elected, but Labour in 2005
gained only 35.2\% and won\ldots

The Electoral Reform Society publishes a list of all the different voting
systems in use (or not, in the case of AV+) in the UK, and where they are
used in other places in the world.

The majority system used in the London Mayor's elections was the
Supplementary Vote(SV).  In this system, voters get two votes: first and
second choice.

Alternative Vote is the most common type of majoritarian system.  Voters
rank the candidates in order of preference---one, two, three, four, five,
and so on---and they are redistributed if neither of the candidates on the
ballot papers get an overall majority, until one does: but the least
preferred candidates are eliminated before this happens.

Alternative Vote Plus is not used anywhere in the world, yet.  It is a
hybrid system composed of AV and an extra, top-up vote.

Single Transferable Vote (STV) is one of the most complex systems in terms
of internal workings.  However, for voters, it is quite simple: they just
rank candidates in order of preference.  STV is used in local and assembly
elections in Northern Ireland, and local elections in Scotland.

In a closed Party List, voters select a party, not the candidates, and so
have no choice over which come first, or last: they are in the order the
party wants them to be in, so they might put better seeming candidates at
the top of the list.  In an open Party List, voters at least have a choice
of preferred candidate.  Party List is a proportional system, but there are
no constutiencies.

Britain remains a two-party system in terms of elections, but a three party
system in terms of votes at elections (if we forget about May 2010 when the
Coalition was formed).

\textsl{(For more clear information, see research work earlier in normal notes,
particularly ticksheet, and all the various tables throughout the
textbook.)}
 
	\subsection*{Definitions}

	\textbf{Elections}: Every five years in the UK.  The people vote for---in principle---who runs the country. 

	\textbf{Majoritarian representation}: Converts votes in an election into seats in government, awarding the most seats to whichever party received the most votes. 

	\textbf{Mandate}: The authority to govern granted to the winning party at an election.  The mandate sugests that the government may implement the measures in its election manifesto.  It also implies that the government has authority to use its judgement in dealing with unforeseen circumstances.

	\textbf{Proportional representation}: Converts votes in an election into seats in government in a broadly proportional way.

	\textbf{Electoral reform}: A process whereby the electoral system is changed or where there is a campaign for such change.

	\textbf{Party system}: The typical structure of parties within a
political system.

	\textbf{Strong and stable government}: A strong government is one that can
confidently lead (say, one that has acheived a clear majority in an
election) and is a party on its own who are stable and have little conflict
internally.

	\subsection*{Other tasks further on background reading notes
sheet}

	\begin{itemize}
		\item{Electoral Reform Society electoral system information
researched as part of in-class research.}
 		\item{I need to find the correct issue of the Politics
Review magazine online at a later date---I couldn't, when I looked
before\ldots}
	\end{itemize}

\end{document}
