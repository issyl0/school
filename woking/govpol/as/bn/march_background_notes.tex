\documentclass[11pt]{article}

\usepackage{parskip}
\usepackage{fullpage}
\usepackage{longtable}

\begin{document}

\section*{Unit  2, section 2.3 (The Prime Minister and Cabinet) background notes}

\subsection*{Notes from textbook}

The typical number of Cabinet members is 23.  The number of senior
non-Cabinet posts is 15.  Junior ministers are not in the Cabinet and are
around 60 in number.  There are on average 17 whips.  The total number of
people in the complete government is around 115, in Westminster.

All members of the government must sit in Parliament.  Most members of the
government, and ministers, are members of the Commons and have
constituencies to look after.  The Prime Minister has one too: David
Cameron is MP for Witney in Oxfordshire.

All members of the government have to be appointed by the Prime Minister,
and can only be dismissed by him\slash her.

Collective cabinet responsibility and individual ministerial resonsibility
are important principles: see the definitions.  An example of individual
ministerial responsibility is David Laws who
resigned from the Coalition government after the expenses scandal: he had
claimed expenses for a flat he lived in with his lover, but said that his
lover was his landlord in order to avoid difficult questions\slash
admitting he was gay.  Ministers resign over their own disagreements in
Cabinet, also, through CCR.

Prime ministerial style is an important factor.  The four main Prime
Ministers in the several previous decades were Margaret Thatcher, John
Major, Tony Blair, Gordon Brown, and now David Cameron.  We can analyse
each of their prime ministerial styles, and have done: see the
balancer\slash reformer\slash innovator and egoist sheet earlier in notes
folder, and the associated research.

There exists a useful table on page 277 of the textbook which for the sake
of brevity I am not going to reproduce here.  It shows the make up of the
2010 Coalition Cabinet.


\begin{longtable}{| p{7cm} | p{7cm} |}
	\hline
	\textbf{Remaining functions of cabinet} & \textbf{Main weaknesses
of cabinet}\\
	\endfirsthead
	\hline
	\textbf{Remaining functions of cabinet (cont.)} & \textbf{Main weaknesses of
cabinet (cont.)}\\
	\endhead
	\hline
	Settling ministerial disputes. & Prime minister is now dominant.\\
	\hline
	Making decisions that cannot be made elsewhere. & Most decisions
are made in committees.\\
	\hline
	Dealing with domestic emergencies. & Meetings are shorter and
stage-managed.\\
	\hline
	Determining presentation of policy. & Large departments have become
more independent.\\
	\hline
	Legitimising decisions made elsewhere. & More decisions are made in
bilateral meetings.\\
	\hline
	Settling coalition disputes. & Much decision making has moved to 10
Downing Street.\\
	\hline
\end{longtable}

A minister without portfolio is a minister who has no fixed role, who is
not responsible for any department.  Baroness Warsi is a minister without
portfolio.

The prime minister's authority and power is questionable.  The Prime
Minister can:
	\begin{itemize}
		\item{Appoint and dismiss ministers.}
		\item{Grant honours.}
		\item{Lead the civil service.}
		\item{Appoint senior judges and bishops.}
		\item{Lead the armed forces.}
		\item{Lead foreign relations.}
		\item{Chair cabinet meetings.}
	\end{itemize}

Limitations on prime ministerial power include the following:

	\begin{itemize}
		\item{The size of a prime minister's parliamentary
majority.}
		\item{Unity of the ruling party, for example frequency of
backbench revolts.}
		\item{Public support and media support and alignment.}
	\end{itemize}

\subsection*{Definitions}

\textbf{Cabinet government}: A system of government where the cabinet is
the central policy-making body.

\textbf{Core executive}: In the United Kingdom, the main body of the
Cabinet---the closest to the Prime Minister.

\textbf{Prime ministerial government}: Political circumstances in which the
prime minister dominates policy making and the whole machinery of
government.

\textbf{Presidentialism}: Where the head of state is the President and a
model of Presidential government operates---big separation of powers, etc. 

\textbf{Political leadership}: The charismatic qualities of a good leader:
his forcefulness and popularity and abilities.  All factors which enhance
the effectiveness of a leader.

\textbf{Collective Cabinet responsibility}: In the UK, all cabinet
decisions must be collectively supported by all members of the government,
at least in public, even if the member concerned disagrees with it.

\textbf{Individual ministerial responsibility}: A step down from CCR.  This
is the principle that Ministers are responsible for the actions of their
department and if civil servants within that department do something wrong,
the Minister him\slash herself should resign as the civil servants cannot
be named as they must remain anonymous, and the Minister should have
correctly advised.

\subsection*{Ministerial resignations since 1997}

See the sheet entitled ``Recent Ministerial Resignations'' in the main
section of my folder.

\end{document} 
