\documentclass[11pt]{article}

\usepackage{parskip}
\usepackage{fullpage}
\usepackage{graphicx}

\begin{document}

\section*{Unit 2, section 2.4 (The judiciary and civil liberties) background
notes}

\subsection*{Notes from textbook}

A long table explaining the functions and make up of the judiciary of
England and Wales is available on page 298 and 299 of the AS textbook.  For
the sake of brevity I have not reproduced it here.

The upper levels of the judiciary and the relationships between them:

\includegraphics[scale-0.5]{}



\subsection*{Definitions}

\textbf{Judicial independence}: The principle that members of the judiciary
should retain independence from any influence by government or parties or
other political movements.

\textbf{Judicial neutrality}: The principle that members of the judiciary
should avoid allowing their political ideas to affet their decisiions in
cases.  It also implies that judges should not show any systematic bias
towards or against any groups in society.

\textbf{Civil liberties}: The rights and freedoms that citizens enjoy in
relation to the state and its laws.  These include, for example, the right
of all to vote or stand for office, freedom of expression, freedom of
association, and the right to a fair trial.

\textbf{Rule of law}: The principle that all citizens are equal under the
law, that all are entitled to a fair trial if accused of a crime and that
the government itsef is subject to law.  In other words, the rule of law
insists that government should not be arbitary but must always act within
the constrainst of the law.

\subsection*{University education of the members of the Supreme Court}

\begin{itemize}
	\item{President of The Supreme Court, The Right Hon the Lord
Phillips of Worth Matravers -- \textsl{University College London}.}
	\item{Deputy President of The Supreme Court, The Right Hon the Lord
Hope of Craighead -- \textsl{University of Cambridge}.}
	\item{Justice of The Supreme Court, The Right Hon the Lord Walker
of Gestingthorpe -- \textsl{University of Cambridge}.}
	\item{Justice of The Supreme Court, The Right Hon the Baroness Hale
of Richmond -- \textsl{University of Cambridge}.}
	\item{Justice of The Supreme Court, The Right Hon the Lord Brown of
Eaton-under-Heywood -- \textsl{University of Oxford}.}
	\item{Justice of The Supreme Court, The Right Hon the Lord Mance --
\textsl{University of Oxford}.}
	\item{Justice of The Supreme Court, The Right Hon the Lord Kerr of
Tonaghmore -- \textsl{Queen’s University, Belfast}.}
	\item{Justice of The Supreme Court, The Right Hon the Lord Clarke
of Stone-cum-Ebony -- \textsl{University of Cambridge}.}
	\item{Justice of The Supreme Court, The Right Honourable Lord Dyson
-- \textsl{University of Oxford}.}
	\item{Justice of The Supreme Court, The Right Hon. Lord Wilson of
Culworth -- \textsl{University of Oxford}.}
	\item{Justice of The Supreme Court, The Right Hon. Lord Sumption --
\textsl{University of Oxford}.}
	\item{Justice of The Supreme Court, The Right Hon. Lord Reed --
\textsl{University of Edinburgh, University of Oxford}.}
\end{itemize}
 
\end{document}
