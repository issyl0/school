\documentclass[a4paper,12pt]{article}

\usepackage{parskip}

\begin{document}

\subsection*{Political participation}

\subsubsection*{Ways in which to participate}

\begin{itemize}
\item{voting}
\item{arguing}
\item{protesting, or rioting}
\item{referendums---example: \textsc{AV}}
\item{local authority or council; voting or becoming a councillor there}
\item{watching the news to keep abreast of what is going on in the world/country and to increase knowledge levels}
\item{paying tax}
\item{opinion polls}
\item{petitions---Govt.\ e-Petitions website}
\item{blogging/social media participation}
\item{becoming a member of a political Party}
\item{debating, even on a small, familial scale}
\item{pressure groups---example: Greenpeace etc.; or, trade union}
\end{itemize}

\textbf{Statistics for declining participation rates:\\ 84\% in 1950.\\ 62\% in 2001.}

\subsubsection*{Reasons for non-activity}

\begin{itemize}
\item{not feeling like it counts}
\item{lack of time}
\item{lack of interest and/or political education}
\item{active in the past, but now not liking how the country is going and feeling powerless to change it because not listened to in the past}
\item{emigration}
\item{immigration---language barriers etc.}
\item{no Party that stands out}
\item{division between MPs and the public---underrepresentation---Statism (many MPs went to Eton, for example)}
\item{content with current Party/system/goings on}
\item{globalisation that limits the importance of one nation?}
\item{Party/class delalignment}
\item{de-industraialisation/de-trade-unionalisation}
\end{itemize}

\subsubsection*{Ways to improve participation}

\begin{itemize}
\item{compulsory voting}
\item{postal votes or e-voting}
\item{electoral reform}
\item{political education}
\end{itemize}

\end{document}
