\documentclass[12pt]{article}

\usepackage{parskip}

\begin{document}

\title{Are UK prime ministers as powerful as it is sometimes claimed?}
\author{Isabell Long}
\maketitle

The extent of UK prime ministerial power is debatable.  Some claim that
UK prime ministers are very powerful, others that they are not so because of
all the scrutiny their Cabinet goes through that they are responsible for. 

In recent years, the prime ministers of the United Kingdom have had their
powers restricted not only by not being popular, but the introduction of
European Union membership and therefore restrictions on parliamentary
sovereignty.  These restrictions have also come about due to the
parliamentary devolution of Scotland, Wales, and Northern Ireland: the
creation of their respective Parliaments or Assemblies.  European Union
membership has entailed that the European Convention on Human Rights be
adhered to, forcing the UK Government to enact the UK-specific Human Rights
Act based heavily on everything in the European Convention on Human Rights,
thereby not allowing them to have their own say and be sovereign over
issues of human rights such as deportation of criminals or immigrants and
arrestation and release, taking away power from the Prime Minister and his
Cabinet (such as Home Secretary Theresa May), an example of which being the recent case
involving the Libyan plane bomber.

The UK prime ministers have always been primus inter parus, or first amongst equals in their
Cabinet, having equal power in Cabinet meetings, but being head of the
government and governing political party, hence ``first''.  Collective
Cabinet Responsibility gives the prime minister more power over his or her
Cabinet---Ministers cannot publicly disagree with what the Cabinet rules
without having to resign.  This rule was specified by a prime minister who
wished to have more power and not have such a divided Cabinet who could
potentially publicly go against him and shame the government.

Internationally, at least in the past, the UK Prime Minister has been
thought of as incredibly powerful, especially compared to the President of
the United States.  For example, the US has a written constitution which
makes new, modern legislation introduction difficult, there is no such
thing as the Royal Prerogative of powers that the President can inherit as
the Prime Minister has done, as the US does not have a Royal family, and
the separation of powers is much greater---the President, for example, is
not allowed to even enter Congress, the US version of the Houses of
Parliament, and must speak to a sympathetic Congressman if he or she wants
to get anything debated.  The Prime Minister's power compared to the
President's has been envied by many a past US President, including one who said that he wished
he ``had as much power as the UK Prime Minister''.  

Overall, whether the Prime Minister is powerful depends on a lot of things
including the checks and balances in place on his or her actions, the
level of scrutiny from the opposition, and as Harold Macmillan said,
``events, [\ldots] events''.

\end{document}
