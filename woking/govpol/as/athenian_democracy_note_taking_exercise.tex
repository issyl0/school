\documentclass[a4paper,12pt]{article}

\usepackage{parskip}

\begin{document}

\textsc{Politics} $=$ ``polis'' e.g. community.  Also ``monarchy'' and ``democracy''.  All Greek words.

Greece had 1500 communities.  ``Demokratica'' $=$ people power.

If $\not=$ democracy then oligarchy in 4th Century \textsc{BC} $=$ power in hands of richest.

`Monarchy' or `tyranny' e.g. power by force.

Stable democracy from Athens; Solon $=$ founder of democracy.

Ephialtes and Pericles liked helping the poor.  Around 400 \textsc{BC} there were oligarchan revolutions which replaced democracy.

The Romans extinguished the Greek's democracy.

The USA and France liked the idea of democracy and used Greece's model.

Men in Athens were paid small amounts of money to compensate for spending time participating politically (jury, etc.) instead of working in the fields or workshops.  These were their duties, but also men's privileges---women, or any non-free citizens (slaves\...) or foreigners were excluded.

\end{document}
