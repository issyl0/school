\documentclass[12pt]{article}

\usepackage{parskip}

\begin{document}
Devolved power is where powers of a main government are granted in part to local or regional government such as the Scottish Parliament.  Federalism is where the powers of the separate states are agreed in the constitution---such as each US state being able to choose whether to impose the death penalty.  Each state cannot have its powers revoked, whereas devolved parliaments' powers can be revoked.

\begin{center}
	\begin{tabular}{| p{5cm} | p{8cm} | }
		\hline
		UK Constitution & US Constitution\\
		\hline
		-- disorganised & -- organised\\
		-- many defined groups & -- laws can be passed and they have to go through a certain process and get sent back if the President doesn't agree\\
		-- uncodified & -- codified\\
		-- Parliament & -- sovereign?\\
		-- not entrenched, flexible & -- relatively inflexibile, entrenched\\
		-- unelected Head of State & -- elected Head of State and upper chamber (right of veto)\\
		-- relative fusion of powers & -- separation of powers---checks and balances\\
		-- bicameral & -- bicameral\\
		\hline
	\end{tabular}
\end{center}

\end{document}