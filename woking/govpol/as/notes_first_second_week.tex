\documentclass[a4paper,12pt]{article}

\usepackage{parskip}

\begin{document}

\textsc{Government}: Typically the members of a governing Party with a paid and/or appointed position.

\textsc{Power}: Ability to make someone do something that they would not otherwise have done --- cohersion.

\textsc{Authority}: Speaking about a government that is legitimate, e.g. in charge.\\ Three types: traditional, charismatic, legal/rational.

\textsc{Legitimacy}: When a government's right to govern is accepted.

\textsc{Citizenship}: Recognising a person's rights and responsibilities as a full member of a particular state.

\textsc{Democracy}: Freedom of vote.

\textsc{Representative democracy}: Everyone has the right to choose who are able to make decisions.

\textsc{Direct democracy}: Every citizen has the right to have a say in decision making.

\textsc{Plutocracy}: Dominated by the rich.\\
\textsc{Oligarchy}: Ruled by the few.
\textsl{(These are opposites of pluralism.)}

\textsc{hoc} and \textsc{hol} $=$ \textsc{hop} $\leftarrow$ bicameral

In a General Election, the voter is fundamentally just voting for his local, constituency MP.  The system is one of \textsc{FPTP}, or First Past the Post.  There are 650 MPs who govern one constituency in Great Britain, including Northern Ireland.

\end{document}
