\documentclass[a4paper,12pt]{article}

\usepackage{parskip}

\begin{document}
	
	\textsc{Subsidance}: Under EU governance, decisions should be made at the lowest possible level.
	
	\textsc{De jure}: We could with draw from the EU.
	
	\textsc{De facto}: Practically not possible.
	
	\textbf{Democratic elements of the UK:}
	
	\begin{itemize}
		\item{elections}
		\item{opponents can question, oppose, and scrutinise}
		\item{freedom of speech}
		\item{representation of the people}
		\item{impartial Civil Service}
		\item{pluralism; multi-party system}
		\item{legitimacy}
		\item{secret ballot}
		\item{opportunities to vote in referendums should there be any}
		\item{universal suffrage (voting)}
		\item{if the government does some thing wrong, it is accountable for its actions---ministerial responsibilty}
	\end{itemize}
	
	\textbf{Undemocratic elements of the UK:}
	
	\begin{itemize}
		\item{unelected House of Lords}
		\item{unelected Cabinet}
		\item{unelected Head of State and Prime Minister}
		\item{secrecy}
		\item{unproportional electoral system}
		\item{party over people---too powerful}
		\item{under 18s cannot vote}
		\item{weak constituency links due to safe seats for MPs in certain constituencies}	
	\end{itemize}
\end{document}