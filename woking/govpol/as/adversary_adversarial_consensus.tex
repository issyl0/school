\documentclass[a4paper,12pt]{article}

\usepackage{parskip}
\begin{document}

Adversary and consensus politics are opposite terms and refer to the political, ideological and policy differences (or lack of) between the major parties.

Adversarial politics refers to the ongoing relationship between HM Government and HM Opposition.

\begin{enumerate}
	\item{Adversary politics is opposing another party's view when there is something to honeslty oppose, whereas adversarial politics is the act of opposing unthinkingly and automatically due to being an opposing and enemy party and wishing to hold the governing party to account.}
	\item{Problems of a confrontational style of politics:
		\begin{itemize}
			\item{voter engagement is lost}
			\item{politicians follow their party and don't say what they think}
			\begin{itemize}
				\item{so, they are in danger of losing their own opinions}
			\end{itemize}
			\item{intolerant}
			\item{opposing ideas for the sake of it}
			\begin{itemize}
				\item{so, the general public will not know exactly where a party stands if they oppose automatically and unthinkingly without believing the oppising point and therefore being sincere about it}
			\end{itemize}
			\item{unconstructive and spiteful}
			\begin{itemize}
				\item{voters are lacking in confidence because of this}
			\end{itemize}
		\end{itemize}
	\item{The key features of consensus politics are that two parties are in agreement over the majority of polices and do not oppose many, if any polices.  Consensus politics is usually used in times of crisis, such as in World War II.}
	\item{The current UK political system is very much adversarial.  In some ways, such as within the Coalition, it is consensual, though only because it has to be---not through any real wish.}
	}
\end{enumerate}
\end{document}
