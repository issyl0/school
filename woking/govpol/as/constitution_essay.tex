\documentclass[11pt,a4paper]{article}

\usepackage{parskip}
\usepackage{fancyhdr}
\setlength{\headheight}{15.2pt}
\pagestyle{fancy}

\lhead{Isabell Long}
\rhead{9th February 2012}

\begin{document}

\textbf{With reference to the source, describe three sources of the UK constitution.}

Three sources of the UK constitution are conventions, common law, and European law.  Conventions are unwritten laws that have come to be through tradition---things that some may say have no explanation as to why they are like they are.  European law is a source for the UK's constitution because the UK is in the European Union and so has to comply with the relevant EU legislation.  Common law is the name for the law developed and adhered to by judges on top of official legislation.

\textbf{With reference to the source and your own knowledge, explain the arguments in favour of a cofied constitution for the UK.}

The arguments in favour of a codified UK constitution include the clear setting of rights of all citizens, it would make the UK line up with other modern democracies, and it would make the UK and its police, courts, and government more organised.

The documentation of rights and responsibilities for citizens would help citizens know their rights and the publication of this documentation would mean that citizens would be able to refer to one document as a definitive guide, not have to go to several different bodies who might give them incorrect or inconsistent information.

Organisation would improve: the governmental decisions would possibly be less haphazard if they knew what they could and could not do and even the conventions could be documented---now they're just seemingly inherently known.

Many other countries who operate in a democracy have a constitution, notably the United States, whose constitution has existed codified for centenaries and continues to work.  This is an example of how constitutions are good and can benefit governments and citizens.  The adoption of a codified constitution would bring the UK in line with these many other democratic countries, and be beneficial to the UK for all the above reasons and more.

\textbf{Make out a case against the adoption of a codified constitution for the UK.}

A constitution is a set of principles according to which a state and citizens are governed.  Constitutions can be either codified or uncodified: written or unwritten.  The UK's is currently unwritten, however many others around the world are documented.

%o UK adopting uncodified beneficial because\\
%	o Inflexible.\\ YES.
%	o Too many conventions.\\ YES.
%	o Would take too long to write.\\ YES.
%	o If it ain't broke, don't fix it.\\ YES.
%o Case against...\\
%	o Inflexibility would help the UK to know where it was going and make a proper plan.\\ YES.
%	o Writing it?  At least it would be written.  Proper amendment process etc.\\
%	o It may well break down in time - think political dissatisfaction and unrest.
	
The UK sticking with an uncodified constitution would be beneficial because codified constitutions are almost completely inflexibile, requiring a proper amendment\slash addition process for any amendments or new additions, which would make it harder to impose laws without an even longer, even more drawn out scrutinisation process.

The number of conventions in the current unwritten constitution is far too great for all of them to be written down---it would take weeks to even explain just their basics.  Conventions are tradition, so by definition probably difficult to explain and justify---they just `are' and `have been' imposed and adhered to.

Another argument against the constitution being codified is that of the fact that due to the amount of laws and regulations, it would take far too long to document.  For example, due to amendments still being able to be passed, amendments would constantly be going through while the previous laws and conventions would still be trying to be understood and put into terms that citizens could understand, not just politicians, so the size of the constitution would be vast.  On the other hand, as a counter argument, it could be that in writing the document, government liase with civil servants and strip out various archaic policies, laws and conventions that have existed for centuaries but are no longer relevant in today's modern world.

The preservation of the autonomy of the judicial review process is another reason to not write everything down---judges would feel that they could use their own judgement less, and would have to reread the constitution document as opposed to just keeping on top of new laws and interpreting the law as they see fit for a particular case, and convention meaning they do this for other cases.  The judiciary and our courts may become too political.

The UK keeping an uncodified constitution would mean that the documentation of the iffy state of the monarchy and its powers would not have be documented clearly and for all to see---there would still be secrecy.  The monarchy has devolved so much power to Parliament and the Prime Minister that some may argue that its existence is pointless and it just exists as a tourist attraction.

As an advantage to codified constitutions in this case, the inflexibility of the long, drawn out amendment and addition process would possibly discourage the UK to be so (arguably) progressive, and not impose so many new laws and amendments to laws.  It would encourage the government to be more open about what it is doing because the constitution would be laid out for every citizen to see, however this in itself is also an argument against it being codified.  The UK has survived for long enough without a codified constitution: as the saying goes, ``if it ain't broke, don't fix it.''

With the amount of political and social unrest in this country---the recent riots and protests being an example---citizens knowing where they stand would curb that and some would say it would make the country happier---but after all that work, it could be that rights help groups still continue to interpret the constitutional document, not taking it as `gospel', so to speak, thus invalidating the efforts: there would still be the possibility of misinforming citizens about their rights, even though the codification would aim to properly inform citizens and ease unrest.

Even though the codified constitution of the United States has existed for centenaries and has worked for America with its federalism and various bodies of law that bills must pass through, that does not mean it would work for every country, especially not the UK in some aspects.  Copying America in every way would not be beneficial to us.

\end{document}