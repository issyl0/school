\documentclass[a4paper]{article}

\usepackage{parskip}
\usepackage{fancyhdr}

\pagestyle{fancy}
\lhead{Isabell Long}
\rhead{5th March 2012}

\begin{document}

\title{Assess the arguments for and against further reform of the House of
Lords.}
\author{Isabell Long}
\maketitle

For centuries there has been much discussion surrounding reforming the House of Lords, but apart from Labour's measures after their election in 1997, where in 1999 they got rid of many hereditary peers, not very much has been done.  After the 2010 General Election, during the campaigns for which every party declared to work towards reform of the House, much noise was generated, and now Nick Clegg, Deputy Prime Minister of the UK, in this Coalition government, has started to draw up concrete plans.

The House of Lords exists to scrutinise laws passed by government, and the workings of the government.  Disrupting the process now, some may argue, may lead to a lot of bills being passed without proper scrutiny---the mess the reforms will cause would potentially leave the House of Lords in limbo while new lords are appointed and procedures and laws to govern the Lords are passed, a potential upset to current practice and an upset to tradition that they might not be willing to accept in the current climate with the current, shaky state of the Coalition Government.

Nick Clegg's reforms, announced in 2012, are an attempt at restarting the reform process started by the Labour party decades ago when they reduced the number of hereditary peers in the House, to 92, via an election of the previously present Lords.  Even after that election, most of the Lords, hereditary or otherwise, do not turn up to the House to vote or sit in on bill debates---some just use their title for their own benefit.

A big case for reforming the House of Lords is that they have always been unelected, therefore unaccountable to the people, and for a chamber that scrutinises and passes legislation, for them not to be accountable and for parties just to push high members into the second chamber on a whim, when they need to win a vote or urgently pass a bill that they do not expect to get through at that time without a few extra votes, is a serious issue.  The Lords are deemed often to be experts in certain fields, to know what they are talking about when it comes to debating and amending legislation, more so than the MPs in the Commons who, despite being elected, may not be necessarily competent or with the times---especially MPs in historically `safe seats'.  Going back to the Lords' unfairness, some argue the age old ``if it ain't broke, don't fix it'' for every aspect of proposed Lords reform, not only this one: it has worked very well for the past decades, despite a slight dwindling where it almost served no purpose just before Labour reformed it, and now it is doing its job, so some might question why the Government would mess that up---some may be sceptic and think that it is a smoke screen for other debates going on while the Lords' reform ones are that are much worse for the country.  For the case against, related to the impracticality of having an elected chamber, the United States often suffers from gridlock: the principle of having the two elected chambers.  In our case, if there was one with a Conservative majority, the other with a Labour majority, there would be a possibility that no bills would get passed because of ideological and policy-based disagreements.

Following on with the ``if it ain't broke, don't fix it'' argument, the House of Lords' scrutinisation and amendment process is very efficient: take the recent welfare bill: it has now been passed, under threat from the Commons of the use of the Parliament Act, but this bill went backwards and forwards to and from both Houses, being discussed and amended.  The House of Lords benefits from having more time to discuss bills, and less pressure to be seen to be dealing with everything all at once, and is not often in the media spotlight, unlike Parliament.  With regard to the Lords' ability to delay the passing of bills, it can do so for up to a year, however after that year the bill must be passed.  Also, related to the Salisbury Convention that is no longer officially applicable due to the 2010 Coalition and them having to make up an agreement that no voter actually voted for, the Lords' cannot refuse to pass a bill that is directly related to something a Party said it would do in its election manifesto, as the people voted for it, and, equally, though not under the same law, they cannot refuse anything to do with finance---hence the controversy surrounding the welfare and NHS bill's benefit caps and reductions, and the bedroom tax. 

An advantage to the House of Lords as it stands now is that there are crossbenchers: Lords who do not belong to any party, who are independent, so in theory as objective as they can be and therefore not affected by potential party whips when it comes to a vote.  Another of Nick Clegg's reforms is to have a partly elected House of Lords, 60\% elected and 40\% unelected: on the one hand, this would be brilliant for democracy as long as the votes for the Lords were held at different times, but on the other, it might be disastrous as every Lord would have to stand for election and the chances of independents getting into power is slim.  In addition, a fully elected or partially elected second chamber may just turn into a glorified, richer, House of Commons, which would not be good for the country: we need some differentiation in order to properly scrutinise, and if the Lords are competing between themselves for seats in the chamber, they may not want to agree or scrutinise legislation quite so carefully when they get into the chamber.  Also, what about voter turnout for Lords elections?  It is highly unlikely that already disillusioned voters (as voter turnout at General Elections shows) want to vote, even at a different time of year that does not coincide with a General Election every fifteen years, for old Lords who they would not think would represent them.  Another important thing to consider is that Commons MPs are split up and given a constituency to represent---what would happen with the Lords?  In practice, there are far too many of them to each represent a corner of the country.  However, to reduce the sheer number of Lords in the country, Nick Clegg is proposing going further than Tony Blair did and cutting the number of Lords in general---not just the hereditaries---to a fixed three hundred.  If this went ahead, there would be fewer Lords than Commons MPs, which, some would argue, would be good for ending quite so much elitism in politics.

The House of Lords has always been populated by usually middle-aged or old, middle class, suited white men.  Of late, the United Kingdom has become obsessed with racial, gender, and religious equality, however these principles wished for in modern life have not made their way up into the political chambers.  Both the House of Commons and the House of Lords have the same percentage of females: 22\%.  This is a shocking figure if we think about the proportion of females in society.  If we consider the underrepresentation of ethnic minorities (their representation is 5\% in the House of Lords), some may argue against the House of Lords' very existence due to it not even beginning to represent society today, so again being ``out of touch''.  This consideration extends to religious representation in the House, too, as bishops are represented (and female bishops are not allowed in the Church of England), but no leaders from other faiths such as Islamic Imams and Jewish Rabbis.  Nick Clegg's reforms suggest that Lords will be required to sit for fifteen years, which is an awfully long time for keeping already retired (most probably), old Lords who are almost certainly out of touch with current goings on in rougher parts of the UK, and this may disadvantage the UK's law making and scrutinising process---at least now there is a relatively good influx of new Lords, albeit very few considerably younger.

Some extremely radical minds may think that the House of Lords needs abolisihing altogether, as as many as one third of democracies around the world work perfectly well with a unicameral system.  Another idea that would restore the Lords' representitiveness is that ``normals'' are invited to be a part of the chamber (but not become Lords because that would boost their egos and they would no longer be ``normals'') to give a more real-world view.  It could be argued that this latter measure would increase voter turnout because voters percieve politicians to be above them, and so having people of the average social class of the non-voter involved may boost morale and lead to more confidence and an increase in votes, if indeed the House of Lords switches to elections.

In conclusion, reform is a hot topic which is not easily resolvable: there are many arguments in favour, and many against.  It remains to be seen how far this Government get, how hard the Lords fight against reform, how many of them come out of the woodwork, but Nick Clegg said in a conference speech laying out his proposals that if there is resistance, ``the will of the Commons will prevail''.

\end{document}
