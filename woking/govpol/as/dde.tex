\documentclass[12pt,a4paper]{article}

\usepackage{parskip}

\begin{document}

\title{Assess the arguments in favour of the greater use of direct democracy in the UK.}
\author{Isabell Long}
\maketitle

In its broadest sense, democracy is simply the people's right to vote and to voice their opinion on matters that they may be concerned about.  Democracy originated in Athens, hence the word's origin: ``demokratica'', the Greek word for ``people power''.  Since its birth, democracy has developed and now exists in many forms.  I will examine these other forms as I assess the arguments for direct democracy.

Like everything, democracy has evolved over the years, so now exists in many forms.  In the United States, there exists a Presidential democracy, however the United Kingdom's system makes use of three types: representative, liberal, and parliamentary.  Representative democracy has a strong emphasis on the fact that, for example, in the United Kingdom MPs are not delegates but representatives---they act in their constituents' best interests instead of relaying their constituents' specific views.  Liberal democracy is defined as tolerant, with free and fair elections with multiple political parties supposedly representing the different sectors of society, and the media and pressure groups, as well as the people, have freedom of speech.  Parliamentary democracy has the underlying principle of being parliamentarily sovereign.  This is the convention that no body can override legislation made by the current Parliament, but the current Parliament of the UK, for example, cannot impose laws that contravene EU legislation.  Also present in a Parliamentary democracy are the core values of all types of democracy: legitimacy, accountability, scrutiny, and representativeness, however there is still a big element of secrecy, as within all things.

Direct democracy, however, is more grass roots.  Its central principle is political participation---people power, ``for the people, by the people''.  Direct democracy was the first form of democracy, so people's methods of participating politically have since evolved.  Now, with there being more people involved and bigger, better structured and established governments, and more ways for the public to get involved, some may argue that with declining participation rates, the public do not take the opportunity to have their say in matters.  To avoid the government being entirely blamed for this, however, the UK government, for example, has introduced many ways of getting involved, some traditional, some new, discovered with the invention of modern technology.  An example of the latter is the UK government's e-Petitions website, where people can put their name down in support of, for example, the building of every new house with solar panels on the roof, and have the government consider debating it if the petition receives one hundred thousand signatures.  Referendums are a more traditional but rarely used form, and referendums are called when the government cannot make a decision on an important subject that affects the people, such as the 2011 Alternative Vote referendum in the UK.  Despite these being only a couple of ways in which one can participate politically, some would argue that Rousseau's famous comment ``the English think they are free, but they are only free once every five years'' is invalid, because although, for example, the British people and people living in democratic countries in general cannot change the actual governing political party more frequently than at a General Election, they can show that they have freedom of speech by participating in other ways than simply going to vote at a General Election, as I have mentioned.

There are pros and cons to direct democracy.  One of direct democracy's benefits is devolution, for example in Britain the Scottish Parliament and Welsh Assembly are devolved from the Parliament in Westminster, to create smaller governments that can do the best for their section of the country due to being focussed solely on that country or part of that country.  It takes time for people to get used to this, however, so making decisions can be slow because not everybody agrees or even knows what they are having their say on.  Direct democracy does, again on the other hand, promote localism, the principle (in politics) of local governmental control, no big governments controlling local issues.  This fits together with the principle of decisions being taken at the lowest possible level.  Also, the government can be held to account through direct democracy---if the people do not like something, they can complain by way of protest for example, as everyone is allowed and expected to take part.

Overall, direct democracy is important and deserves to be pushed to the forefront, as political participation needs to be encouraged.  The UK cannot, however, revert to the Athenian model as that would give too much power to the people and it, in its current state, needs defined leaders.

\end{document}
