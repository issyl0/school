\documentclass[11pt, a4paper]{article}

\usepackage{parskip}
\usepackage{fancyhdr}
\setlength{\headheight}{15.2pt}
\pagestyle{fancy}

\lhead{Isabell Long}
\rhead{25th January 2012}

\begin{document}

\textbf{How do pressure groups promote functional representation?}

Pressure groups promote functional representation by representing a cause,
or a section of society, with regard to particular issue.  They often
campaign around a very small issue, such as the League of Cruel Sports
which campaigns against animal abuse and the existence of cruel sports such
as fox hunting.  As a result, society feels more in touch with a specific
issue and people can choose which granular issues to support rather than
the gigantic political parties which deal with everything from health to
education.

\textbf{Explain three factors which may restrict the influence of a
pressure group.}

Three factors potentially restricting the influence of a pressure group
are funds, membership, and position.  In order to function, pressure groups need to have money to carry out duties and campaign.  In order to have money, they need to either have rich donors, or many members, or a high membership charge, or all three of these things.  In order to gain a strong following and eventual membership, the pressure groups need a decent, accessible cause.  The more money they have, the more influence they can have by engaging in campaigns and potentially lobbying MPs and Parliament.

A point has to be made also about insider and outsider pressure groups.
Generally it is thought that insider groups have more influence, as they
are inside governmental walls, so to speak, and have risen to that
position through a potential multitude of ways, whereas outsider groups
are less likely to influence government.  However, this distinction is not
much of a distinction, because groups drop in to and out of favour
depending on the governing political party of the time, and some outsider
groups have managed to be successful, such as Friends of the Earth who
successfully mounted an appeal against the solar feed-in-tariff rate
reductions.
 
\textbf{Are pressure groups becoming more powerful, or less powerful?}

The number of pressure groups that exist in the UK is quite substantial.
As a result, their powers as a form of expression in general, and
individually, can be called into question.

Pressure groups allow the people to have a greater say in issues that affect them, be these issues traditional (animal rights with the \textsc{RSPCA}), or more obscure such as \textsc{CAMRA}, the Campaign for Real Ale.  Pressure groups do not have defined leaders, as such: being not as restricted as political parties in their movements, their members quite often support everything the group does and get involved.  Therefore, they are powerful if members actively participate.

Another helper for pressure group success is the insider--outsider status distinction.  This distinction is not always clear, however it is usually said that insider pressure groups do have more influence, and therefore power, than outsider groups.  However, the insider and outsider status of a particular pressure group may change depending on whether a particular group needs to be insider for the government to hear its views, depending on the political party that is in power.  Some pressure groups become redundant when the political parties already support and have in their manifestos and plans for governing what the pressure group campaigns for, and when the political parties do not, or the leadership\slash direction of the Party changes, a particular pressure group may need to become active again and push for change, if they do not have the particular policy implemented.  This applies mainly to insider pressure groups, however outsider pressure groups may not need to be so forceful in terms of gaining insider status or lobbying if the Party already recognises their cause.

Lobbying is another way in which pressure groups can be seen as powerful.  Lobbying is paying for influence, basically.  Lobbying can cost billions, so it requires a very monetarily powerful pressure group to carry it out, or special lobbying firms.  There have been scandals surrounding lobbying, however: paying MPs personally to ask questions is one notable one.  This has called into question the ethics and, indeed, the efficiency of lobbying.  Some may argue that why shouldn't MPs ask certain questions anyway, if these questions have been raised by pressure groups but have not been asked due to the MP not thinking them worthy, even when from the general public, who MPs are supposed to represent, or these questions have been asked at constuency surgeries by concerned or interested constituents.

The choice of pressure groups has appealed to the electorate, with pressure group membership numbers rising while political Party membership numbers dwindling over the last few years, according to various figures.

A dampener on success of various pressure groups is the negative publicity
that---having no defined leaders or accountability---they are prone to.
Greenpeace became quite anarchic when they took over boats to oppose whale
fishing, hanging signs and apparently kidnapping people, and Fathers for
Justice's members, the rights for fathers group, some of whom dressed up as
batman and scaled buildings, again extreme measures that painted them in a
bad light in the media.

In conclusion, pressure groups have broadly become more powerful in terms of membership numbers and the number of issues represented by all of the groups.

\end{document}

